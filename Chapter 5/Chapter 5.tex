\documentclass{article}

\usepackage[left=1.5cm, right=1.5cm, top=3cm, bottom = 3cm]{geometry}

\usepackage{amsmath}
\usepackage{amsfonts}
\usepackage{amssymb}
\usepackage{graphicx}
\usepackage{float}
\usepackage{wrapfig}
\usepackage{latexsym}
\usepackage{hyperref}
\usepackage{feynmf}
\usepackage{exscale}
\usepackage{relsize}
\linespread{1.1}
\newcommand{\tr}{\mathop{\mathrm{Tr}}}
\usepackage{braket}

%%%%%%%
%第五章习题安排:
%%%%%%%
%宋志坚 1
%宋盛雨央 2
%陈博文 3
%解放 4
%辜晨曦 5
%鲍亦澄 6
%蒋文韬 
%李嘉琛 
%颜公望 
%张传坤 7
%王志凌 8

\author{SM-at-THU}
\title{\bf{Solutions to Pathria's Statistical Mechanics}\\Chapter 5}

\begin{document}
\maketitle
\section*{Problem 5.1}

\section*{Problem 5.3}
(a) Take the trace in momentum space. $\hat{H} = \hat{p}^{2}/2m$ so the density matrix is
\begin{equation}
\hat{\rho} = \frac{e^{-\beta \hat{H}}}{\tr (e^{-\beta \hat{H}})} = (\frac{m}{2\pi \beta \hbar^{2}})^{-3/2}e^{-\frac{\beta}{2m} \hat{p}^{2} }
\end{equation}
A simple observation is $\braket{\textbf{p'}|\hat{\rho}|\textbf{p}} \sim \delta_{\textbf{p}\textbf{p'}}$, which means without interaction the particle will propagate as a plane wave.\\
(b) Use Fourier transformation to get the density matrix in momentum space, i.e.
\begin{eqnarray}
\rho(k',k) &=& \sqrt{\frac{m \omega}{2 \pi \hbar \sinh (\beta \hbar \omega)}}\int dx dx' \exp \left\{
- \frac{m \omega}{2 \hbar \sinh (\beta \hbar \omega)} \left[(x'^{2} +x^{2}) \cosh (\beta \hbar \omega) -2 x' x \right] + i(k'x' - kx)
 \right\} \notag \\
&=& \sqrt{\frac{m \omega}{2 \pi \hbar \sinh (\beta \hbar \omega)}} \sqrt{\frac{4 \pi \hbar \sinh (\beta \hbar \omega)}{m \omega (\cosh \beta \hbar \omega -1 )}} \sqrt{\frac{4 \pi \hbar \sinh (\beta \hbar \omega)}{m \omega (\cosh \beta \hbar \omega +1 )}} \notag \\
&\times& \exp \left\{-\left[ \frac{(\frac{k'-k}{2})^{2} \hbar \sinh \beta \hbar \omega}{m \omega (\cosh \beta \hbar \omega -1)} + \frac{(\frac{k'+k}{2})^{2} \hbar \sinh \beta \hbar \omega}{m \omega (\cosh \beta \hbar \omega +1)}
 \right] \right\} \notag \\
&=& \sqrt{\frac{8 \pi \hbar }{m \omega \sinh (\beta \hbar \omega)}} \exp \left\{ - \frac{\hbar\left[ (k'^{2}+k^{2}) \cosh \beta \hbar \omega - 2k'k\right]}{2 m \omega \sinh \beta \hbar \omega}
\right\}
\end{eqnarray}
The expression resembles the real space density matrix. This is easy to comprehend that momentum and coordinate are conjugate in the SHO case. Other properties are just like the cases in \emph{Section 5.3}.






\section*{Problem 5.4}
If we take the unsymmetrized wave function
$$
\psi_E(\mathbf{q}) = \prod_{i=1}^N u_{\epsilon_i}(\mathbf{q}_i)
$$
then the density matrix will be:
\begin{equation}
\langle 1,\cdots,N|e^{-\beta H}|1',\cdots,N'\rangle = \sum_{E}e^{-\beta E}\psi_E(\mathbf{r}_1,\cdots,\mathbf{r}_N)\psi^*_E(\mathbf{r}_1',\cdots,\mathbf{r}_N')=\sum_{E}e^{-\beta E}\prod_{i=1}^N u_{\epsilon_i}(\mathbf{r}_i)u^*_{\epsilon_i}(\mathbf{r}_i)
\end{equation}
Since we are considering about an ideal classical gas, the energy eigenstates are also momentum eigenstates, so we can find that the eigenfunction of momentum will be:
$$
u_{\mathbf{k}_i} = \frac{1}{\sqrt{V}}e^{i\mathbf{k}\cdot\mathrm{r}}
$$
we can easily find that the density matrix will be:
\begin{equation}
\langle 1,\cdots,N|e^{-\beta H}|1',\cdots,N'\rangle = \sum_{\mathbf{k}_1,\cdots \mathbf{k}_N}e^{-\frac{\beta \hbar^2}{2m}(\mathbf{k}^2_1+\cdots+\mathbf{k}_N^2)}\prod_{i=1}^N\frac{1}{V}e^{i\mathbf{k}_i\cdot(\mathbf{r}_i-\mathbf{r}_i')}
\end{equation}
so the result will be
\begin{equation}
\langle 1,\cdots,N|e^{-\beta H}|1',\cdots,N'\rangle = \prod_{i=1}^N\frac{1}{(2\pi)^3}\int d^3k \,e^{-\frac{\beta \hbar^2}{2m}\mathbf{k}_i^2+i\mathbf{k}_i\cdot(\mathbf{r}_i-\mathbf{r}_i')} = \prod_{i=1}^N\left(\frac{2\pi m}{\beta h^2}\right)^{3/2}\exp\left[-\frac{m}{2\beta\hbar^2}(\mathbf{r}_i-\mathbf{r}_i')^2\right]
\end{equation}
Obviously there is no correlation term between different particles. Then calculate the partition function:
\begin{equation}
Q_N = \left(\frac{2\pi m}{\beta h^2}\right)^{3N/2}\int \prod_{i=1}^N d^3r_i = V^N \left(\frac{2\pi m}{\beta h^2}\right)^{3N/2}
\end{equation}
We can find that there is no Gibbs’ correction factor in the partition function.


\section*{Problem 5.5} % (fold)
\label{sec:problem_5_5}
	
	The partition function of the noninteracting,indistinguishable particles system is:
	$$Q_N(V,T)=\frac{1}{N!\lambda^{3N}}\sum_P \delta_P[f(Pr_1-r_1)\cdots f(Pr_N-r_N)]$$
	where $f(r)=e^{-\frac{\pi r^2}{\lambda^2}}$
	The first approximation is:
	$$\sum_P=1\pm\sum_{i<j}f_{ij}f_{ji}$$
	So the partition function in first approximation is:
	$$Q_N(V,T)=\frac{1}{N!\lambda^{3N}}\int(1\pm\sum_{i<j}e^{-\frac{\pi r_{ij}^2}{\lambda^2}})d^{3N}r$$
% section problem_5_5 (end)

\end{document}