\documentclass{article}

\usepackage[left=2.5cm, right=2.5cm, top=3cm, bottom = 3cm]{geometry}

\usepackage{amsmath,bm}
\usepackage{amsfonts}
\usepackage{amssymb}
\usepackage{graphicx}
\usepackage{float}
\usepackage{wrapfig}
\usepackage{latexsym}
\usepackage{hyperref}
\usepackage{feynmf}
\usepackage{exscale}
\usepackage{relsize}
\usepackage{listings}
\usepackage{upquote}


\linespread{1.1}


\author{\bf{Gongwang Yan}\footnote{yan-gw13@mails.tsinghua.edu.cn}
\\
\\ \emph{Department of Physics, Student No.2013012215}  }
\date{}
\title{\bf{Homework No.2 for Cryptography}}
\bibliographystyle{unsrt}
\begin{document}
\maketitle
\section*{2.15}
\begin{figure}[htbp]
  \includegraphics[width=0.650\textwidth]{pic215.jpg}
\end{figure}
\section*{2.16}
\paragraph{(a)}

\paragraph{(b)}

\section*{2.17}
\paragraph{(a)}
$v-k_0=s-ru-k_0=r(b+k_1-k_1-a)=r(b-a)$, because p is a prime and $r\neq 0(mod\ p)$, so $v-k_0=0\Leftrightarrow a=b(mod\ p)$, and we have $p>n$,so $a=b(mod\ p)\Leftrightarrow a=b$.
\paragraph{(b)}
Alice knows $a,k_0,k_1,r(b-a)$, Carol knows $r,k_0+r(b-a),k_1+a$. When $k_0,k_1$ independently run all over the $Z_p$ uniformly, $k_0+r(b-a)$ distributes all over $Z_p$ uniformly independent of r(b-a), $k_1+a$ distributes all over $Z_p$ uniformly independent of a, so Carol gets no information about a, b. When $b-a\neq 0$, r(b-a) distributes all over $Z^*_p$ uniformly independent of (b-a) when r runs all over $Z^*_p$ uniformly, so Alice gets no more information other than $a\neq b$.
\section*{2.22}
\paragraph{(a)}
B wins when A wins in both the first stage and the second stage or A loses in both stages. So $Pr[B\ wins]=(1/2+\epsilon)^2+(1/2-\epsilon)^2=1/2+2\epsilon^2$.
\paragraph{(b)}
We want to show that there is an adversary who has winning probability larger than 1/2 no matter what the security parameter $\lambda$ is. In the given arguement, if A's winning probability runs up and down around 1/2 as $\lambda$ changes, neither A nor B meets the requirement because both of their winning probabilities fluctuate around 1/2.
\paragraph{(c)}

\end{document}
