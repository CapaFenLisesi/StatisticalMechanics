\documentclass{article}

\usepackage[left=1.5cm, right=1.5cm, top=3cm, bottom = 3cm]{geometry}

\usepackage{amsmath}
\usepackage{amsfonts}
\usepackage{amssymb}
\usepackage{graphicx}
\usepackage{float}
\usepackage{wrapfig}
\usepackage{latexsym}
\usepackage{hyperref}
\usepackage{feynmf}
\usepackage{exscale}
\usepackage{relsize}
\usepackage{bm}%bold math, for vector
\linespread{1.1}

%%%%%%%
%第六章习题安排:
%%%%%%%
%宋志坚 1,2,3
%宋盛雨央 11,21,31
%陈博文 12,22,32
%解放 13,23,33
%辜晨曦 4,14,24,34
%鲍亦澄 5,15,25
%蒋文韬 6,16,26
%李嘉琛 7,17,27
%颜公望 8,18,28
%张传坤 9,19,29
%王志凌 10,20,30

\author{SM-at-THU}
\title{\bf{Solutions to Pathria's Statistical Mechanics}\\Chapter 6}

\begin{document}
\maketitle
\section*{Problem 6.1}


\section*{Problem 6.4}
According to the definition of S:
$$S=-k\sum_{r}P_{r}ln(P_{r})$$
So the density of electrical is $P_{r}$
$$dU=\sum_{r}P_rdE_r$$
Because we consider the interaction potential energy twice.
$$\phi(r)=\phi_{ext}(r)+e\int\frac{n(r')}{|r-r'|}dr'$$
According to gauss's law:
$$n(r)=\epsilon\nabla^2{\phi}$$


\section*{Problem 6.6} % (fold)
\label{sec:problem_6_6}
	For any speed distribution, we have $ P(u)\ge0,u\ge 0 $. Utilize the Cauchy inequality:
	\begin{align*}
		&|\bm a|\cdot|\bm b|\ge \bm a \cdot \bm b\\
		\therefore \left< u\right> \left< \frac{1}{u} \right>&=\sum_u P(u)u \sum_u P(u)\frac{1}{u}\\
		&=\sum_u \left(\sqrt{P(u)u}\right)^2 \sum_u \left(\sqrt{P(u)  \frac{1}{u} }\right)^2\\
		&\ge \left( \sum_u \sqrt{P(u)u} \cdot \sqrt{P(u)  \frac{1}{u}} \right)^2\\
		&=\left( \sum_u P(u) \right)^2=1
	\end{align*}
	$ \therefore $ proved.

	In addition, it's easy to verify that for Maxwellian distribution,
	\begin{equation}
		\left< u\right> \left< \frac{1}{u} \right>=\frac{4}{\pi}
	\end{equation}
% section problem_6_6 (end)



\section*{Problem 6.8}
\begin{eqnarray*}
Z&=&\frac{1}{N!}(\int e^{-\beta (gz+\frac{p^2}{2m})}\frac{d^3\vec{p}d^3\vec{x}}{h^3})^N  \\
&=&\frac{1}{N!}(\frac{S(2\pi m)^{3/2}(1-e^{\beta gL})}{\beta^{5/2}h^3 g})^N
\end{eqnarray*}
Where S is the cylinder's cross area. The rest calculation is of little value. Note that we can naturally infer that the heat capacity is larger than that of free space situation by arguing that higher temperature will flatten the z-distribution and raise the gravitational potential energy.


\section*{Problem 6.11}
\paragraph*{(a)}
\begin{equation}
\int_{0}^{\infty} C \mathrm{e}^{-\beta c (p^2+m_0^2c^2)^{1/2}} p^2 dp = Cm_0^3c^3 \int_0^{\infty} \mathrm{e}^{-\beta m_0c^2 (y^2+1)^{\frac{1}{2}}} y^2 dy = Cm_0^3c^3 \int_0^{\infty} \mathrm{e}^{-\beta m_0c^2 \cosh t} \sinh^2t \cosh t dt
\end{equation}

\begin{equation}
\sinh^2 t \cosh t = \frac{1}{4} (\cosh 3t -\cosh t)
\end{equation}
As
\begin{equation}
K_{\alpha}(x)= \int_0^{\infty} \mathrm{e}^{-x\cosh t} \cosh \alpha t dt  
\end{equation}
So,
\begin{equation}
\int_{0}^{\infty} C \mathrm{e}^{-\beta c (p^2+m_0^2c^2)^{1/2}} p^2 dp = \frac{Cm_0^3c^3}{4} (K_3(\beta m_0c^2)-K_1(\beta m_0c^2)) = C \frac{m_0^2 c}{\beta} K_2(\beta m_0 c^2)
\end{equation}

\begin{equation}
C=\frac{\beta}{m_0^2 c K_2(\beta m_0 c^2)}
\end{equation}


\paragraph*{(b)}
When $\beta m_0c^2 \gg 1$,$\frac{1}{K_2(\beta m_0c^2)}=e^{\beta m_0c^2} \sqrt{\frac{2\beta m_0c^2}{\pi}}$.
\begin{equation}
C=4\pi (\frac{\beta}{2\pi m_0})^{\frac{3}{2}}e^{\beta m_0c^2}
\end{equation}
Because $\beta m_0 c^2 \gg 1$, so the $p$ with significant probability is much smaller than $m_0c$,so
\begin{equation}
\mathrm{e}^{-\beta c (p^2+m_0^2c^2)^{1/2}} = e^{-\beta m_0c^2(1+\frac{p^2}{2m_0^2c^2})}
\end{equation}
And
\begin{equation}
f(p) = (\frac{\beta}{2\pi m_0})^{\frac{3}{2}} e^{-\beta \frac{p^2}{2m_0}} 4\pi p^2
\end{equation}

When $\beta m_0c^2 \ll 1$,$\frac{1}{K_2(\beta m_0c^2)}= \frac{\beta^2 m_0^2 c^4}{2}$.
\begin{equation}
C=\frac{\beta^3 c^3}{2}
\end{equation}
\begin{equation}
\mathrm{e}^{-\beta c (p^2+m_0^2c^2)^{1/2}} = e^{-\beta pc}
\end{equation}
So,
\begin{equation}
f(p)=\frac{\beta^3 c^3}{8 \pi} 4\pi p^2 
\end{equation}

\paragraph*{(c)}
We denote $y=\frac{p}{m_0c}$
\begin{equation}
u=(\frac{y^2}{1+y^2})^{\frac{1}{2}} c 
\end{equation}
So,
\begin{equation}
\langle pu \rangle = C m_0^4c^5 \int_0^{\infty} \mathrm{e}^{-\beta m_0c^2 (y^2+1)^{\frac{1}{2}}} (\frac{y^2}{1+y^2})^{\frac{1}{2}} y^3 dy =  C m_0^4c^5 \int_0^{\infty} \mathrm{e}^{-\beta m_0c^2 \cosh t} \sinh^4 t dt
\end{equation}
\begin{equation}
\sinh^4t = \frac{3\cosh t + \cosh 4t - 4\cosh 2t}{8}
\end{equation}
So,
\begin{equation}
\langle pu \rangle = C m_0^4c^5 \frac{3K_0(\beta m_0c^2) + K_4(\beta m_0c^2)  - 4K_2(\beta m_0c^2)}{8}
\end{equation}
Because
\begin{equation}
K_4(x)+3K_0(x)-4K_2(x)= \frac{24K_2(x)}{x^2}
\end{equation}
Then,
\begin{equation}
\langle pu \rangle = \frac{\beta}{m_0^2 c K_2(\beta m_0 c^2)} m_0^4c^5 \frac{3K_2(\beta m_0 c^2)}{\beta^2 m_0^2 c^4} = 3kT
\end{equation}

\section*{Problem 6.13}
For classical ideal gases, the velocity distribution of the atoms are the Maxwell distribution, which can be obtained by canonical distribution:
\begin{equation}
f(\mathbf{u}) = \left(\frac{m}{2\pi kT}\right)^{3/2}e^{-\frac{m\mathbf{u}^2}{2kT}}
\end{equation}
\paragraph{a)}
So the number of impacts made by gas molecules on a unit area of the wall in a unit time for which the angle of incidence lies between $\theta$ and $\theta + d\theta$ will be:
\begin{equation}
\frac{dN}{dAdt} = \frac{N}{V}\int_0^\infty du\int_0^{2\pi} d\phi \left(\frac{m}{2\pi kT}\right)^{3/2}e^{-\frac{m\mathbf{u}^2}{2kT}}u^3\sin{\theta} \cos{\theta}d\theta=\frac{N}{V}\sqrt{\frac{2kT}{\pi m}}\sin\theta\cos\theta d\theta
\end{equation}
\paragraph{b)}The number of impact for which the speed lies between $u$ and $u + du$ is
\begin{equation}
\frac{dN}{dAdt} = \frac{N}{V}\int_0^{2\pi}d\phi \int_0^{\pi/2}d\theta \left(\frac{m}{2\pi kT}\right)^{3/2}e^{-\frac{m\mathbf{u}^2}{2kT}}u^3\sin{\theta} \cos{\theta}du=\frac{\pi N}{V} \left(\frac{m}{2\pi kT}\right)^{3/2}e^{-\frac{m\mathbf{u}^2}{2kT}}u^3 du
\end{equation}
\paragraph{c)}
The rate of dissociate is given by the following impact number with following condition $u> \sqrt{2\epsilon_0/m} = u_0$ in which $\epsilon_0 = 10^{-19}\mathrm{J}$, so the impacting number will be:
\begin{eqnarray}
\frac{dN}{dAdt} &=& \frac{N}{V}\int_0^{2\pi}d\phi \int_0^{\pi/2}d\theta \int_{u_0}^\infty \left(\frac{m}{2\pi kT}\right)^{3/2}e^{-\frac{m\mathbf{u}^2}{2kT}}u^3\sin{\theta} \cos{\theta}\nonumber\\
&=&\frac{\pi N}{V}\int_{u_0}^\infty du\,u^3\left(\frac{\epsilon_0}{u_0^2 kT}\right)^{3/2}\exp{\left(-\frac{\epsilon_0}{u_0^2 kT}u^2\right)}
\end{eqnarray}
Then define a function $f(u,T)$ as shown:
$$
f(u,T) = \left(\frac{\epsilon_0}{u_0^2 kT}u^2\right)^{3/2}\exp{\left(-\frac{\epsilon_0}{u_0^2 kT}u^2\right)}
$$
so for two different temperature, we can calculate the ratio of the function $f(u,T)$ :
\begin{eqnarray}
\frac{f(u,T_2 = 310\mathrm{K})}{f(u,T_1 = 300\mathrm{K})}& =& \left(\frac{T_1}{T_2}\right)^{3/2}\exp\left[\frac{\epsilon_0 u^2}{u_0^2k}\left(\frac{1}{T_1}-\frac{1}{T_2}\right)\right]\nonumber\\
&\geq&\left(\frac{T_1}{T_2}\right)^{3/2}\exp\left[\frac{\epsilon_0}{k}\left(\frac{1}{T_1}-\frac{1}{T_2}\right)\right]=2.075> 2
\end{eqnarray}
so the integral of $f(u,T_2)$ will twice larger than that of $f(u,T_1)$, which means the dissociation probability will be more than twice.


\section*{Problem 6.14}
Consider the $v_x$ and the $v_y$ are independent 
$$E_x+E_y=kT$$
We can find the $E_z$ equal to the single molecular's pressure to he wall.
$$E_z=kT$$ 
So the average energy is 2kT


\section*{Problem 6.16} % (fold)
\label{sec:problem_6_16}
	According to eqt 6.4.13, the effusion rate for Boltzmanian gas is
	\begin{equation}
		R=\frac{1}{4}n\left< u \right>= \frac{n}{4}\sqrt{\frac{8}{\pi} \frac{kT}{m}}
	\end{equation}

	From kinetic consideration, $ R \propto \frac{P \Delta S}{m} $. Therefore, for both A and B side, $ P \propto \sqrt{mT} $, i.e.,
	\begin{equation}
		\frac{P_A}{P_B}=\sqrt{\frac{m_AT_A}{m_BT_B}}
	\end{equation}
	
% section problem_6_16 (end)


\section*{Problem 6.18}
\begin{eqnarray*}
\bar{v}&=&\frac{\int e^{-a v^2}v d^3\vec{v}}{\int e^{-a v^2} d^3\vec{v}}  \\
\bar{v}_{respective}&=&\frac{\int e^{-a (v^2_1+v^2_2)}|\vec{v_1}-\vec{v_2}| d^3\vec{v_1}d^3\vec{v_2}}{\int e^{-a (v^2_1+v^2_2)} d^3\vec{v_1}d^3\vec{v_2}}  \\
&=&\frac{\int e^{-\frac{a}{2} ((\vec{v_1}-\vec{v_2})^2+(\vec{v_1}+\vec{v_2})^2)}|\vec{v_1}-\vec{v_2}| d^3(\vec{v_1}-\vec{v_2})d^3(\vec{v_1}+\vec{v_2})}{\int e^{-\frac{a}{2} ((\vec{v_1}-\vec{v_2})^2+(\vec{v_1}+\vec{v_2})^2)} d^3(\vec{v_1}-\vec{v_2})d^3(\vec{v_1}+\vec{v_2})}  \\
&=&\frac{\int e^{-\frac{a}{2} (\vec{v_1}-\vec{v_2})^2}|\vec{v_1}-\vec{v_2}| d^3(\vec{v_1}-\vec{v_2})}{\int e^{-\frac{a}{2} (\vec{v_1}-\vec{v_2})^2} d^3(\vec{v_1}-\vec{v_2})}  \\
&=&\frac{\int e^{-a v^2}\sqrt{2}v d^3\vec{v}}{\int e^{-a v^2} d^3\vec{v}}  \\
&=&\sqrt{2}\bar{v}
\end{eqnarray*}
Proved.

\section*{Problem 6.21}
As temperatures is high enough to allow classical approximation for the rotational motion of the molecules.\\
For a heteronuclear molecules,
\begin{equation}
j_{rot}(T)=\sum_0^{\infty} (2l+1) \exp(-l(l+1) \frac{\hbar^2}{2IkT}) \approx \int_0^{\infty} (2l+1) \exp(-l(l+1) \frac{\hbar^2}{2IkT}) dl =\frac{T}{\Theta_r}
\end{equation}
For a homonuclear molecules,
\begin{equation}
j_{rot}(T) = \frac{T}{2\Theta_r}
\end{equation}
Here,$I=\mu r_0^2$,$\Theta_r=\frac{\hbar^2}{2Ik}$.
So,
\begin{equation}
\mu_{int} = -kT \mathrm{ln}j_{rot}
\end{equation}
\begin{equation}
\mu_{trans}= kT \mathrm{ln}(\frac{N}{V} \left(\frac{\hbar^2}{2\pi mkT}\right)^{\frac{3}{2}} )
\end{equation}
\begin{equation}
K[T]=\exp(-\beta(2\mu_{HD}-\mu_{H_2}-\mu_{D_2})) =\frac{\left [\exp(-\beta \epsilon_{0(HD)}) \frac{\frac{T}{\Theta_{r(HD)}}}{n_0 \lambda_{HD}^3} \right]^2} {\left [\exp(-\beta \epsilon_{0(H_2)}) \frac{\frac{T}{2\Theta_{r(H_2)}}}{n_0 \lambda_{H_2}^3} \right] \left [\exp(-\beta \epsilon_{0(D_2)}) \frac{\frac{T}{2\Theta_{r(D_2)}}}{n_0 \lambda_{D_2}^3} \right]}
\end{equation}
We can simplify it as
\begin{equation}
K[T]=4 \exp(-\beta(2\epsilon_{0(HD)}-\epsilon_{0(H_2)}-\epsilon_{0(D_2)})) \frac{\Theta_{r(D_2)}\Theta_{r(H_2)}}{\Theta_{r(HD)}^2} \frac{\lambda_{H_2}^3\lambda_{D_2}^3}{\lambda_{HD}^6}
\end{equation}
\begin{equation}
K[T]=4 \exp(-\beta(2\epsilon_{0(HD)}-\epsilon_{0(H_2)}-\epsilon_{0(D_2)})) \frac{I_{HD}^2}{I_{H_2}I_{D_2}} \frac{m_{HD}^3}{\sqrt{m_{H_2}^3m_{D_2}^3}}
\end{equation}
We can assume that $\frac{I_{HD}^2}{I_{H_2}I_{D_2}} \approx 1$, $\frac{m_{HD}^3}{\sqrt{m_{H_2}^3m_{D_2}^3}} \approx 1$. When $\beta \ll 1$ ,we have $K[\infty] = 4$



\section*{Problem 6.23}
The first step is to find the equilibrium point:
$$
\frac{\partial V(r)}{\partial r} = 0\quad\Rightarrow\quad r_{\mathrm{eq}} = r_0 
$$
Then assume that the oscillating mode will not be excited at the lowest energy scale of rotation mode, and calculate the moment of inertia:
$$
I = \frac{mr_0^2}{2}
$$
in which $m$ is the mass of a hydrogen atom and $r_0$ is the equilibrium length. So the energy scale of the rotation mode will be
\begin{equation}
\epsilon_{\mathrm{rot}} = \frac{\hbar^2}{mr_0^2}l(l+1)
\end{equation}
and the characteristic temperature of rotation will be
$$
T_{\mathrm{rot}} = \frac{\hbar^2}{kmr_0^2} \sim 78\mathrm{K}
$$
Then talk about the oscillation mode. Expand the potential energy around $r_0$ and we can find the potential will be:
\begin{equation}
V(r)\sim -V_0 + \frac{1}{2}\frac{2V_0}{a^2}(\delta r)^2
\end{equation}
since the reduced mass of the hydrogen molecule is $\mu = m/2$, the oscillating frequency is
\begin{equation}
\omega = \sqrt{\frac{2V_0}{a^2\mu}} = \sqrt{\frac{4V_0}{a^2 m}}
\end{equation}
and its characteristic temperature is
$$
T_{\mathrm{oc}} = \frac{\hbar}{k}\sqrt{\frac{4V_0}{a^2 m}}\sim 6394\mathrm{K}
$$
Since $T_{\mathrm{oc}}\gg T_\mathrm{rot}$, the approximation that the molecule is rigid when we calculate the moment of inertia is acceptable. And our conclusion is that the oscillating mode will begin to contribute to the heat capacity at temperature $\sim 6000\mathrm{K}$ and vibration mode will begin to contribute to heat capacity at temperature $\sim 70\mathrm{K}$.



\section*{Problem 6.24}
As the change of the r,the potential energy also be changed:
$$\frac{\delta E}{E}=\frac{\delta r}{r}=\frac{\delta L}{L}^2$$
$$\delta L=J(J+1)\hbar^2$$
$$L=I\omega$$
so we can get the equation 
$$\frac{\delta r}{r}=\frac{\hbar}{\mu r^2\omega}^2(J+1)J$$

\section*{Problem 6.26} % (fold)
\label{sec:problem_6_26}
	
% section problem_6_26 (end)


\section*{Problem 6.28}
Again, I regard this problem as a pure calculational exercise and have no interest to do it now. (Actually up till now my problems are all such things, so I will preserve those seemingly more tedious work. )



\section*{Problem 6.31}
$T=1500K$
\begin{equation}
K[T] = \exp(-\beta(2\mu^{0}_{CO_2}-\mu^{0}_{O_2}-2\mu^{0}_{CO})) = 4.2 \times 10^{10}
\end{equation}
So,$[CO]/[CO_2] = 4.9 \times 10^{-5}$ for the case of $[O_2] = 0.01$\\
$T = 600 K$
\begin{equation}
K[T] = \exp(-\beta(2\mu^{0}_{CO_2}-\mu^{0}_{O_2}-2\mu^{0}_{CO})) = 1.7 \times 10^{40}
\end{equation}
So,$[CO]/[CO_2] = 7.6 \times 10^{-20}$ for the case of $[O_2] = 0.01$



\section*{Problem 6.33}
Assumes that the initial density of $\mathrm{O}_2$ and $\mathrm{CH}_4$ is 
\begin{eqnarray*}
n(\mathrm{O}_2) &=& n_{O_2}n_0\\
n(\mathrm{CH}_4)&=& n_{CH_4}n_0
\end{eqnarray*}
then the equilibrium equation tells us that the remained density of methane will be:
\begin{equation}
\delta = n_0 \frac{4n_{CH_4}^3}{(n_{O_2}-2n_{CH_4})K(T)}
\end{equation}
And from the given condition, we can calculate the equilibrium constant will be:
\begin{equation}
K(T) = \exp\left(-\beta \Delta \mu^{(0)}\right) = 6.75\times 10^{27}\,.
\end{equation}


\section*{Program 6.34}
We know the Saha equation is
$$\frac{\alpha^2}{1-\alpha^2}=\frac{2.4*10^{-4}}{p}T^2.5e^{-\frac{E}{kT}}$$
$\alpha$ is density of ion/density of molecular.So you put the $E=5.139eV$ into the equation you can get the relationship and plot.Plot has been ignored.
\end{document}
