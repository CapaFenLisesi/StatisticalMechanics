\documentclass{article}

\usepackage[left=1.5cm, right=1.5cm, top=3cm, bottom = 3cm]{geometry}

\usepackage{amsmath}
\usepackage{amsfonts}
\usepackage{amssymb}
\usepackage{graphicx}
\usepackage{float}
\usepackage{indentfirst}
\usepackage{wrapfig}
\usepackage{latexsym}
\usepackage{hyperref}
\usepackage{feynmf}
\linespread{1.1}

\author{SM-at-THU}
\title{\bf{Solutions to Pathria's Statistical Mechanics}}

\begin{document}
\maketitle
\section*{Problem 1.1}

\section*{Problem 1.2}
Utilizing the addictive characteristic of $S=f(\Omega)$ and get
\begin{align}
&S=S_1+S_2=f(\Omega_1)+f(\Omega_2)\\
&(\frac{dS}{d\Omega_1})_{\Omega_2}=f'(\Omega_1)\\
&(\frac{dS}{d\Omega_2})_{\Omega_1}=f'(\Omega_2)\\
\end{align}

Inspect a small pertubation near the equilibrium state using the fact that $S=f(\Omega)=f(\Omega_1\Omega_2)$
\begin{equation}
(\frac{dS}{d\Omega_1})_{\Omega_2}=\lim_{\Delta\rightarrow 0}{\frac{f((\Omega_1+\Delta)\Omega_2)-f(\Omega_1\Omega_2)}{\Delta}}
\end{equation}
Assume that $\delta=\Delta\Omega_2$
\begin{equation}
(\frac{dS}{d\Omega_1})_{\Omega_2}=\lim_{\Delta\rightarrow 0}{\Omega_2\frac{f(\Omega_1\Omega_2+\Delta\Omega_2)-f(\Omega_1\Omega_2)}{\Delta\Omega_2}}=\lim_{\delta\rightarrow 0}{\Omega_2\frac{f(\Omega+\delta)-f(\Omega)}{\delta}}=\Omega_2f'(\Omega)
\end{equation}
Apply to $(\frac{dS}{d\Omega_2})_{\Omega_1}$, we can get similar result.
\begin{equation}
(\frac{dS}{d\Omega_2})_{\Omega_1}=\Omega_1f'(\Omega)
\end{equation}
Finally,
\begin{align}
f'(\Omega_1)&=\Omega_2f'(\Omega)=\frac{\Omega_2}{\Omega_1}f'(\Omega_2)\\
\Omega_1f'(\Omega_1)&=\Omega_2f'(\Omega_2)\\
\end{align}
It is obvious that this equation holds for all $\Omega$. Set the value of the equation constant k. 
\begin{align}
\Omega\frac{df(\Omega)}{d\Omega}&=k\\
f(\Omega)&=kln\Omega+C\\
\end{align}
Using a special value $\Omega=1$
\begin{align}
f(\Omega*1)&=f(\Omega)+f(1)\\
C=&f(1)=0\\
\end{align}
And get the result 
\begin{equation}
S=f(\Omega)=kln\Omega
\end{equation}
\section*{Problem 1.3}
\section*{Problem 2.3}
The Hamiltonian of a rotator in 2 dimension is:
\begin{equation}
H = \frac{L^2}{2I}
\end{equation}
in which $L$ is the angular momentum and $I$ is the moment of inertia.
\section*{Problem 2.4}
If we just consider about the orbital angular momentum, it can be written as a function of $p_\theta$ and $p_\phi$ which are the canonical momentum conjugate to the spherical coordinate variables $\theta$ and $\phi$: 
\begin{equation}
L^2 = p_\theta^2 + \frac{p_\phi^2}{\sin^2 \theta}
\end{equation}
thus the phase volume of the region which satisfies $L^2 \leq M^2$ is
\begin{eqnarray}
\bf{\Omega}& = &\int_0^\pi d\theta \int_0^{2\pi} d\phi \int_{L^2\leq M^2} dp_\theta dp_\phi\nonumber\\
&=& \int_0^\pi d\theta \int_0^{2\pi} d\phi \, \pi  M^2\sin\theta\nonumber\\
&=& 4\pi^2 M^2
\end{eqnarray}
Thus the number of microstates is $\Omega = {\bf{\Omega}}/h^2 = M^2/\hbar^2$. Then let us calculate the number by quantized angular momentum. The number can be written as the following form:
\begin{equation}
\Omega = \sum_{j=0}^{j_{\mathrm{max}}}(2j+1) = (j_{\mathrm{max}}+1)^2
\end{equation}
Now we have to determine the number $j_{\mathrm{max}}$. Since we want the absolute value of the angular momentum $< M^2$, we can find that $j_{\mathrm{max}}$ is determined by the following equation:
\begin{equation}
j_{\mathrm{max}} = \Big{\lfloor} \frac{\sqrt{1+\frac{4M^2}{\hbar^2}}-1}{2}\Big{\rfloor}
\end{equation}


\section*{Problem 2.5}
In this problem we need to use the WKB approximation in Quantum Mechanics.

\end{document}