\documentclass{article}

\usepackage[left=1.5cm, right=1.5cm, top=3cm, bottom = 3cm]{geometry}

\usepackage{amsmath}
\usepackage{amsfonts}
\usepackage{amssymb}
\usepackage{graphicx}
\usepackage{float}
\usepackage{wrapfig}
\usepackage{latexsym}
\usepackage{hyperref}
\usepackage{feynmf}
\usepackage{exscale}
\usepackage{relsize}
\linespread{1.1}

%%%%%%%
%第四章习题安排:
%%%%%%%
%宋志坚 1,2,
%宋盛雨央 10,20
%陈博文 3,13
%解放 4,14
%辜晨曦 5,15
%鲍亦澄 6,16
%蒋文韬 7,17
%李嘉琛 8,18
%颜公望 9,19
%张传坤 11
%王志凌 12

\author{SM-at-THU}
\title{\bf{Solutions to Pathria's Statistical Mechanics}\\Chapter 4}

\begin{document}
\maketitle
\section*{Problem 4.1}

\section*{Problem 4.4}
The probability of a state with energy $E_r$ and particle number $N$ is
\begin{equation}
p_{r,N} = \frac{e^{-\beta E_{r,N}+\beta \mu N}}{\mathcal{Q}(\mu,V,\beta)}
\end{equation}
in which $\mathcal{Q} = \sum_{r,N}e^{-\beta E_{r,N}+\beta \mu N}$ is the grand canonical partition function. Since we have define that $z = e^{\beta\mu}$, the probability can be written as:
\begin{equation}
p_{r,N} = \frac{z^N e^{-\beta E_{r,N}}}{\mathcal{Q}(z,V,\beta)}
\end{equation}
So the probability that has exactly N particles will be:
\begin{equation}\label{prob}
p_N = \sum_r p_{r,N} = \frac{z^N\sum_{r}e^{-\beta E_{r,N}}}{\mathcal{Q}(z,V,\beta)}
\end{equation}
easily we can find the summation in the numerator is the canonical partition function of system with $V,N$ and $\beta$:
\begin{equation}
Q_N(V,\beta) = \sum_{r}e^{-\beta E_{r,N}}
\end{equation}
Thus Eq.(\ref{prob}) will become: 
\begin{equation}
p_N = \frac{z^N Q_N(V,\beta)}{\mathcal{Q}(z,V,\beta)}
\end{equation}
For ideal classical gas, the canonical partition function is:
\begin{equation}
Q_N(V,T) = \frac{V^N}{N!}\left(\frac{2\pi mkT}{h^2}\right)^{3N/2}
\end{equation}
and the grand partition function is
\begin{equation}
\mathcal{Q}(z,V,\beta) = \sum_{N = 0}^\infty z^N Q_N(V,T) = \exp\left[zV\left(\frac{2\pi mkT}{h^2}\right)^{3/2}\right]
\end{equation}
Clearly the probability distribution of particle number is
\begin{equation}
p_N = \frac{1}{N!}\frac{(zV\lambda_T^{-3})^N}{e^{zV\lambda_T^{-3}}}
\end{equation}
It is obvious that this distribution is a Poison distribution. From the knowledge of Poison distribution, we know the root-mean-square value of $(\Delta N)$ is
\begin{equation}
\Delta N = \sqrt{zV\lambda_T^{-3}}=\sqrt{e^{\beta\mu}V\left(\frac{2\pi mkT}{h^2}\right)^{3/2}}
\end{equation}
We can also get this result from the formula of grand canonical ensemble:
\begin{eqnarray}
\Delta N &=&kT\sqrt{ \left(\frac{\partial^2 \ln\mathcal{Q}}{\partial\mu^2}\right)_{T,V}}\nonumber\\
&=&\sqrt{e^{\beta\mu}V\left(\frac{2\pi mkT}{h^2}\right)^{3/2}}
\end{eqnarray}
And this result is consistent with the one we get by Poison distribution.

\section*{Problem 4.5} % (fold)
\label{sec:problem_4_5}
	
	We could know from 4.3.20:
	$$S=kT(\frac{\partial q}{\partial T})_{z,V}-Nkln(z)+kq$$
	We can know partial differential:
	$$(\frac{\partial q}{\partial T})_{z,V}-(\frac{\partial q}{\partial T})_{\mu,V}=(\frac{\partial q}{\partial z})_{T,V}(\frac{\partial z}{\partial T})_{\mu,V}$$
	$$(\frac{\partial q}{\partial z})_{T,V}=\frac{N}{z}$$
	So we can infer that:
	$$S=k[\frac{\partial (Tq)}{\partial{T}}]_{V,\mu}$$
% section problem_4_5 (end)
\section*{Problem 4.9}
This problem is totally identical with Mr.Ni's material for class, I regard it meaningless to move those calculations here.

\section*{Problem 4.12}
\begin{equation}
	\overline{NE}=\frac{\sum NE e^{- \alpha N- \beta E}}{\sum e^{-\alpha N-\beta E}}=\frac{\frac{\partial}{\partial \alpha}(\frac{\partial}{\partial \beta}\Xi)}{\Xi}=\frac{\frac{\partial}{\partial \alpha}(\Xi \frac{\partial}{\partial \beta}\ln \Xi)}{\Xi}=\frac{1}{\Xi} \frac{\partial \Xi}{\partial \alpha}\frac{\partial}{\partial \beta}\ln \Xi +\frac{\partial}{\partial \alpha}(\frac{\partial}{\partial \beta}\ln \Xi)
\end{equation}
	While the first part can be written as
\begin{equation}
	\frac{1}{\Xi} \frac{\partial \Xi}{\partial \alpha}\frac{\partial}{\partial \beta}\ln \Xi= \frac{\partial \ln \Xi}{\partial \alpha}\frac{\partial}{\partial \beta}\ln \Xi=\overline{N}* \overline{E}
\end{equation}
	So the equation equals
\begin{equation}
	\overline{NE}-\overline{N}* \overline{E}=\frac{\partial}{\partial \alpha}(\frac{\partial}{\partial \beta}\ln \Xi)=-\frac{\partial}{\partial \alpha} U=-\frac{\partial U}{\partial N}\frac{\partial N}{\partial \alpha}= (\frac{\partial U}{\partial N})\overline{(\Delta N)^2}
\end{equation}

\section*{Problem 4.14}
The Clausius–Clapeyron equation is
$$
\frac{dP_\sigma}{dT} = \frac{L}{T\Delta v}
$$
Since the volume of liquid is negligible compared to that of gas, we can alternate $\Delta v$ by $v_g = kT/P_\sigma$. Put all of these into the Clausius–Clapeyron equation, we can get a differential equation:
\begin{equation}
\frac{dP_\sigma}{P_\sigma} = \frac{L}{R}\frac{dT}{T^2}
\end{equation}
so the solution to the differential equation will be:
\begin{equation}\label{cce}
P_\sigma(T) = P_0\exp\left[\frac{L}{R}\left(\frac{1}{T_0}-\frac{1}{T}\right)\right]
\end{equation}
From the problem we know that $L = 2260\mathrm{kJ/kg} = 40680\mathrm{J/mol}$, $T_0 =373\mathrm{K}$ and $P_0 = 101\mathrm{kPa}$. Then we put all these numbers into Eq.(\ref{cce}) and we get the equilibrium vapor pressure is
$$
P_\sigma(473\mathrm{K}) = 1619\mathrm{kPa}
$$
Experiment result is $P_\sigma \sim 1500\mathrm{kPa}$, and our calculation is approximately correct.

\section*{Problem 4.15} % (fold)
\label{sec:problem_4_15}
	

	According to Clausius-Clapeyron equation.And ignore the volume of solid phase.
	$$\frac{dP_{\sigma}}{dT}=\frac{L}{TV}$$
	Use the gas equation.
	$$ln(p)=-\frac{L}{kT}+A$$
	Use the triple point parameter.
	$$ln(p)=-\frac{L}{kT}+6.6\times10^{26}$$
% section problem_4_15 (end)
\section*{Problem 4.19}
According to the thermodynamic relation:
$$
0=d(G-\mu N)=-Nd\mu +Vdp-SdT  
$$
We have:
\begin{eqnarray*}
d\mu_1 &=& \frac{V_1}{N_1}dp-\frac{S_1}{N_1}dT  \\
d\mu_2 &=& \frac{V_2}{N_2}dp-\frac{S_2}{N_2}dT 
\end{eqnarray*}
On the coexisting curve, $d\mu_1=d\mu_2$. So,
$$\frac{dp_{\sigma}}{dT}=\frac{s_B-s_A}{v_B-v_A}$$
Proved.
\end{document}
