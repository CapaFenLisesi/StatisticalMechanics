\documentclass{article}

\usepackage[left=1.5cm, right=1.5cm, top=3cm, bottom = 3cm]{geometry}

\usepackage{amsmath}
\usepackage{amsfonts}
\usepackage{amssymb}
\usepackage{graphicx}
\usepackage{float}
\usepackage{wrapfig}
\usepackage{latexsym}
\usepackage{hyperref}
\usepackage{feynmf}
\usepackage{enumerate}
\linespread{1.1}

\author{SM-at-THU}
\title{\bf{Solutions to Pathria's Statistical Mechanics}\\Chapter 1}

\begin{document}
\maketitle
\section*{Problem 1.1}
\begin{equation}
\ln\Omega^{(0)}(E^{(0)},E_1)=\ln\Omega(E_1)=\ln(\Omega_1(E_1)\Omega_2(E_2))
\end{equation}
We can expand $\ln\Omega_1(E_1)$ and $\ln\Omega_2(E_1)$ into series near $\bar{E_1}$, and the leading terms are
\begin{align}
&\ln\Omega_1(E_1)=\ln\Omega_1(\bar{E_1})+\left. \frac{\partial \Omega_1(E_1)}{\partial E_1}\right|_{E_1=\bar{E_1}}(E_1-\bar{E_1})+\frac{1}{2}\left. \frac{\partial^2 \Omega_1(E_1)}{\partial E_1^2}\right|_{E_1=\bar{E_1}}(E_1-\bar{E_1})^2+\cdots\\
&\ln\Omega_2(E_1)=\ln\Omega_2(\bar{E_1})+\left. \frac{\partial \Omega_2(E_1)}{\partial E_1}\right|_{E_1=\bar{E_1}}(E_1-\bar{E_1})+\frac{1}{2}\left. \frac{\partial^2 \Omega_2(E_1)}{\partial E_1^2}\right|_{E_1=\bar{E_1}}(E_1-\bar{E_1})^2+\cdots\\
\end{align}
We can set the complicated derivatives to simple symbols
\begin{align}
&\left. \frac{\partial \Omega_1(E_1)}{\partial E_1}\right|_{E_1=\bar{E_1}}=a_1\\
&\left. \frac{\partial \Omega_2(E_1)}{\partial E_1}\right|_{E_1=\bar{E_1}}=a_2\\
&\left. \frac{\partial^2 \Omega_1(E_1)}{\partial E_1^2}\right|_{E_1=\bar{E_1}}=b_1\\
&\left. \frac{\partial^2 \Omega_2(E_1)}{\partial E_1^2}\right|_{E_1=\bar{E_1}}=b_2\\
\end{align}
\begin{align}
&\ln\Omega(E_1)=\ln(\Omega_1(\bar{E_1})\Omega_2(\bar{E_1}))+(a_1+a_2)(E_1-\bar{E_1})+\frac{b_1+b_2}{2}(E_1-\bar{E_1})^2+\cdots
\end{align}
This fuction reach its maximum under thermodynamic equilibrium condition at $E_1=\bar{E_1}$. Thus the linear term must vanish.
\begin{align}
\ln\Omega(E_1)&=\ln(\Omega_1(\bar{E_1})\Omega_2(\bar{E_1}))+\frac{b_1+b_2}{2}(E_1-\bar{E_1})^2+\cdots\\
&\approx \frac{b_1+b_2}{2}(E_1-\bar{E_1})^2+\ln(\Omega(\bar{E_1}))\\
\end{align}
\begin{align}
\Omega(E_1)&=e^{(b_1+b_2)(E_1-\bar{E_1})^2+\ln(\Omega(\bar{E_1}))}=Ae^{\frac{b_1+b_2}{2}(E_1-\bar{E_1})^2}\\
A&=e^{\ln(\Omega(\bar{E_1}))}\\
\end{align}
Which is obviously a gaussian function.\\
Gaussian RMS can be easily determined to be
\begin{align}
\frac{1}{2(b_1+b_2)}=\frac{1}{2}\frac{1}{(\frac{\partial \beta_1}{\partial E_1}+\frac{\partial \beta_2}{\partial E_2})}=\frac{1}{2}\frac{1}{\frac{1}{k{T_1}^2C_{v1}}+\frac{1}{k{T_2}^2C_{v2}}}
\end{align}
For the example of ideal classical gases, we can subtitute $C_vi=\frac{3}{2}N_ik$ and obtain $\frac{3}{2}k^2T^2\frac{N_1N_2}{N_1+N_2}$\\

\section*{Problem 1.2}
Utilizing the additive characteristic of $S=f(\Omega)$ and get
\begin{align}
&S=S_1+S_2=f(\Omega_1)+f(\Omega_2)\\
&(\frac{dS}{d\Omega_1})_{\Omega_2}=f'(\Omega_1)\\
&(\frac{dS}{d\Omega_2})_{\Omega_1}=f'(\Omega_2)
\end{align}
Inspect a small pertubation near the equilibrium state using the fact that $S=f(\Omega)=f(\Omega_1\Omega_2)$
\begin{equation}
(\frac{dS}{d\Omega_1})_{\Omega_2}=\lim_{\Delta\rightarrow 0}{\frac{f((\Omega_1+\Delta)\Omega_2)-f(\Omega_1\Omega_2)}{\Delta}}
\end{equation}
Assume that $\delta=\Delta\Omega_2$
\begin{equation}
(\frac{dS}{d\Omega_1})_{\Omega_2}=\lim_{\Delta\rightarrow 0}{\Omega_2\frac{f(\Omega_1\Omega_2+\Delta\Omega_2)-f(\Omega_1\Omega_2)}{\Delta\Omega_2}}=\lim_{\delta\rightarrow 0}{\Omega_2\frac{f(\Omega+\delta)-f(\Omega)}{\delta}}=\Omega_2f'(\Omega)
\end{equation}
Apply to $(\frac{dS}{d\Omega_2})_{\Omega_1}$, we can get similar result.
\begin{equation}
(\frac{dS}{d\Omega_2})_{\Omega_1}=\Omega_1f'(\Omega)
\end{equation}
Finally,
\begin{align}
f'(\Omega_1)&=\Omega_2f'(\Omega)=\frac{\Omega_2}{\Omega_1}f'(\Omega_2)\\
\Omega_1f'(\Omega_1)&=\Omega_2f'(\Omega_2)\\
\end{align}
It is obvious that this equation holds for all $\Omega$. Set the value of the equation constant k.
\begin{align}
\Omega\frac{df(\Omega)}{d\Omega}&=k\\
f(\Omega)&=kln\Omega+C\\
\end{align}
Using a special value $\Omega=1$
\begin{align}
f(\Omega*1)&=f(\Omega)+f(1)\\
C=&f(1)=0\\
\end{align}
And get the result
\begin{equation}
S=f(\Omega)=k\ln\Omega
\end{equation}
\section*{Problem 1.3}
When the two systems are brought together, they can form an isolated system. Energy and particle number are constant while entropy will not not decreasing in such a system.\\
\begin{align}
E_A+E_B&=E_0\\
N_A+N_B&=N_0\\
dS_A+dS_B&\geq0\\
\end{align}
Apply derivation and get
\begin{align}
dE_A+dE_B&=0\\
dN_A+dN_B&=0\\
dS_A+dS_B&\geq0\\
\end{align}
Subtitute these relations into equation
\begin{align}
dE_A&=T_AdS_A-p_AdV_A+\mu_A dN_A\\
dE_B&=T_BdS_B-p_BdV_B+\mu_B dN_B\\
dV_A&=0\\
dV_B&=0\\
\end{align}
and get
\begin{align}
\frac{dE_A}{dN_A}\geq\frac{\mu_A T_B-\mu_B T_A}{T_B-T_A}
\end{align}

\section*{Problem 1.4}
Suppose $N$ is the number of particles, $v_{0}$ is the volume occupied by one particle and therefore the total number of microstates $\Omega$ is
\begin{equation}
\Omega = \frac{1}{N!}(\frac{V}{v_{0}}) \dots (\frac{V}{v_{0}}-N+1)
\end{equation}
Following $(1.4.2)$, we have
\begin{eqnarray}
\frac{P}{T} &=& k \left(\frac{\partial \ln \Omega}{\partial V}\right)_{N,E} \\
&=& k \frac{\partial \Omega}{\Omega \partial V} \\
&=& k \frac{N}{V} \left(1+ \frac{(N-1)v_{0}}{2V} + \dots \right)
\end{eqnarray}
Considering only the first two terms, it corresponds to $P(V-b) = NkT$ with $b=N v_{0}/2$. \\
Notes: I don't know why the problem says $b=4N v_{0}$ since this gas is hard sphere gas. Anyone has an idea?


\section*{Problem 1.5}
Using equation $(A.11)$, and setting $K = \pi \sqrt{\varepsilon}/L$, it is straight forward to achieve
\begin{equation}
\Sigma_{1}(\varepsilon) = \frac{\pi}{6}\varepsilon^{3/2} \pm \frac{3 \pi}{8} \varepsilon
\end{equation}
where the first term is the volume term ($V=L^{3}$) and the next one is the surface correction ($S=6L^{2}$).



\section*{Problem 1.6}
Use the formula for ideal gas $PV=NkT$.
\begin{equation}
N k \times 300= 10^{5} \times \frac{\pi}{10}
\end{equation}
Thus $\Delta T = 10^{4}/Nk \sim 955 K$.

\section*{Problem 1.10}
    Just use equation $(1.4.21)$ and $(1.4.23)$, we have:
    \begin{equation}
        S(N,V,E)=Nk\ln \left[V\left(\frac{2\pi m k T}{h^2}\right)^{3/2}\right]+\frac{3}{2}Nk
    \end{equation}
    Since He and Ar have the same N,V. We can get the T that He and Ar have the same entropy:
    $$
        T=0K (?)
    $$
\section*{Problem 1.11}
    As $N_2$ and $O_2$ are mixed together at the same pressure and temperature, we can know that the volume of mixed gas is: $V=V_1+V_2$. And we can get the entropy of mixing by utilizing equation $(1.5.3)$:
    \begin{equation}
        \Delta S=k \left[N_1 \ln\frac{V}{V_1}+N_2 \ln\frac{V}{V_2}\right]
    \end{equation}
    for per mole of the air formed:
    \begin{equation}
        \begin{aligned}
            \Delta S_n&=k \left[N_1 \ln\frac{V}{V_1}+N_2 \ln\frac{V}{V_2}\right]/(n_1+n_2)\\
                    &=R \left[n_1 \ln\frac{V}{V_1}+n_2 \ln\frac{V}{V_2}\right]/(n_1+n_2)\\
                    &=4.16~J\cdot mol^{-1}\cdot K^{-1}\\
        \end{aligned}
    \end{equation}
\section*{Problem 1.12}

    \begin{enumerate}[(a)]
        \item   Equation $(1.5.3a)$ can be written as: 
                \begin{equation}
                \begin{aligned}
                    \left(\Delta S \right)_{1\equiv 2}&=N_1\ln \frac{(V_1+V_2)N_1}{V_1(N_1+N_2)}+N_2\ln \frac{(V_1+V_2)N_2}{V_2(N_1+N_2)}\\
                                                    &=(N_1+N_2)\left[y\ln \frac{y}{x}+(1-y)\ln\frac{1-y}{1-x}\right]
                \end{aligned}
                \end{equation}
                Here $x=V_1/(V_1+V_2), y=N_1/(N_1+N_2)$.\\
                
                Consider the function $f(x,y)=y \ln \frac{y}{x}$, we can get the second derivatives:
                \begin{equation}
                    D^2 f(x,y)=\left[
                    \begin{array}{cc}
                        y/x^2 & -1/x \\
                        -1/x & 1/y \\
                    \end{array}\right]
                \end{equation}
                Since $D^2 f(x,y)$ is a positive-semidefinite, $f(x,y)$ is a convex function. Then we can know that:
                \begin{equation}
                    \frac{1}{2} f(x,y)+\frac{1}{2}f(1-x,1-y)\geq f(1/2,1/2)=0
                \end{equation}
                This means $\left(\Delta S \right)_{1\equiv 2}\geq 0$ and the equality holding only when $N_1/V_1=N_2/V_2$
                
        \item   Suppose that $N=N_1+N_2$. And we have $\left(\Delta S \right)^*$ by utilizing equation $(1.5.4)$ :
                \begin{equation}
                \begin{aligned}
                    \left(\Delta S \right)^*&=k\left[N_1 \ln \frac{N}{N_1}+N_2 \ln \frac{N}{N_2} \right]\\
                                            &=k\left[N \ln N- N_1 \ln N_1 -N_2 \ln N_2 \right]
                \end{aligned}
                \end{equation}
                Then we have the derivative of $\left(\Delta S \right)^*$ with respect to $N_1$:
                \begin{equation}
                \begin{aligned}
                    \frac{d \left(\Delta S \right)^*}{d N_1}&=-\ln N_1-\frac{\partial N_2}{\partial N_1}\ln N_2\\
                                                            &=-\left(\ln N_1-\ln N_2\right)
                \end{aligned}
                \end{equation}
                It shows that $\frac{d \left(\Delta S \right)^*}{d N_1}$ satisfies:
                \begin{equation}
                \frac{d \left(\Delta S \right)^*}{d N_1}
                    \left\{
                \begin{aligned}
                    &<0 \qquad  N_1>N_2\\
                    &=0 \qquad  N_1=N_2\\
                    &>0 \qquad  N_1<N_2\\
                \end{aligned}
                    \right.
                \end{equation}
                So we can know that $\left(\Delta S \right)^*$ have the only maximum value at $N_1=N_2$:
                \begin{equation}
                    \max \left(\Delta S \right)^*=\left(N_1+N_2\ln 2\right)
                \end{equation}
                Then we get:
                \begin{equation}
                    \max \left(\Delta S \right)^*\leq\left(N_1+N_2\ln 2\right)
                \end{equation}
                The equality holding when and only when $N_2=N_2$
    \end{enumerate}

\section*{Problem 1.16}
Theorem:\\
If $f(x,y,z)=0$,then we have
\[\left(\frac{\partial x}{\partial y}\right)_z \left(\frac{\partial y}{\partial x}\right)_z =1\]
\[\left(\frac{\partial x}{\partial y}\right)_z \left(\frac{\partial y}{\partial z}\right)_x \left(\frac{\partial z}{\partial z}\right)_y=-1\]
\paragraph{(a)}
\[\frac{S}{N}=-\left(\frac{\partial \mu }{\partial T}\right)_{P}\]
\[\frac{V}{N}=\left(\frac{\partial \mu }{\partial P}\right)_{T}\]
\[\frac{S}{V}=-\frac{\left(\frac{\partial \mu }{\partial T}\right)_{P}}{\left(\frac{\partial \mu }{\partial P}\right)_{T}}=
-\frac{1}{\left(\frac{\partial T}{\partial \mu }\right)_{P} \left(\frac{\partial \mu }{\partial P}\right)_{T}}=
\left(\frac{\partial P}{\partial T}\right)_{\mu }\]
\paragraph{(b)}
\[\frac{V}{N}=\left(\frac{\partial \mu }{\partial P}\right)_{T}\]
\[V \left (\frac{\partial P}{\partial \mu} \right )_{T}=N\]

\end{document}
