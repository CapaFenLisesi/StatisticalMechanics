\documentclass{article}

\usepackage[left=1.5cm, right=1.5cm, top=3cm, bottom = 3cm]{geometry}

\usepackage{amsmath,bm}
\usepackage{amsfonts}
\usepackage{amssymb}
\usepackage{graphicx}
\usepackage{float}
\usepackage{wrapfig}
\usepackage{latexsym}
\usepackage{hyperref}
\usepackage{feynmf}
\usepackage{exscale}
\usepackage{relsize}
\usepackage{listings}
\usepackage{upquote}
\linespread{1.1}

%%%%%%%
%第三章习题安排:
%%%%%%%
%宋志坚 1,2,3,4
%宋盛雨央 10,20,30,40
%陈博文 11,21,31,41
%解放 12,22,32,42
%辜晨曦 13,23,33,43
%鲍亦澄 14,24,34,44
%蒋文韬 5,15,25,35
%李嘉琛 6,16,26,36
%颜公望 7,17,27,37
%张传坤 8,18,28,38
%王志凌 9,19,29,39

\author{SM-at-THU}
\title{\bf{Solutions to Pathria's Statistical Mechanics}\\Chapter 3}

\begin{document}
\maketitle


\section*{Problem 3.1}
In fact the solution to this problem is just a mathematical derivation with only little physics.
\subsection*{(a)}
    \begin{align}
    \mathcal{LHS}&=\langle \left( \Delta n_r \right)^2 \rangle  \\
    &=\langle n_r^2 \rangle +\langle n_r \rangle^2\\
    &=\left. \frac{1}{\Gamma}\left( \omega_r \frac{\partial}{\partial \omega_r } \right)^2\Gamma \right|_{\omega_r=1,\forall r}
    - \left. \left(   \omega_r \frac{\partial}{\partial \omega_r }  \left(\ln \Gamma \right)  \right)^2 \right|_{\omega_r=1,\forall r}\\
    &= \left. \frac{1}{\Gamma} \left( \omega_r \frac{\partial}{\partial \omega_r } + \omega_r^2 \frac{\partial^2}{\partial \omega_r^2} \right)\Gamma \right|_{\omega_r=1,\forall r}
    - \left. \left( \frac{1}{\Gamma}   \omega_r \frac{\partial}{\partial \omega_r } \Gamma \right)^2 \right|_{\omega_r=1,\forall r}\\
    \mathcal{RHS}&=\left. \left( \omega_r \frac{\partial}{\partial \omega_r } \right)^2\left(\ln \Gamma \right)  \right|_{\omega_r=1,\forall r}\\
    &= \left. \frac{1}{\Gamma}  \omega_r \frac{\partial}{\partial \omega_r }\Gamma \right|_{\omega_r=1,\forall r}
    - \left. \left( \frac{1}{\Gamma}   \omega_r \frac{\partial}{\partial \omega_r } \Gamma \right)^2 \right|_{\omega_r=1,\forall r}
    +\left. \frac{1}{\Gamma}\omega_r^2 \frac{\partial^2}{\partial \omega_r^2} \Gamma \right|_{\omega_r=1,\forall r}\\
    &=\mathcal{LHS}
    \end{align}
\subsection*{(b-1)}
    \begin{align}
    & U=\frac{\sum_r\omega_r E_r \exp{(-\beta E_r)}}{\sum_r\omega_r \exp{(-\beta E_r)}}\\
    \Rightarrow& \frac{\partial\beta}{\partial\omega_r}= \frac{(E_r-U)\exp{(-\beta E_r)}} {\sum_r\omega_r (E_r-U) E_r \exp{(-\beta E_r)}}
    \end{align}

    \begin{align}
    \mathcal{LHS}&=\frac{\partial\beta}{\partial\omega_r}
    = \frac{(E_r-U)\exp{(-\beta E_r)}} {\sum_r\omega_r (E_r-U) E_r \exp{(-\beta E_r)}}\\
    &= \frac{(E_r-U)\exp{(-\beta E_r)}  /  \sum_r\omega_r \exp{(-\beta E_r)}} {\sum_r\omega_r (E_r-U) E_r \exp{(-\beta E_r)}  /  \sum_r\omega_r \exp{(-\beta E_r)}}\\
    &= \frac{E_r-U} {\left< E_r^2 \right> -\left< E_r \right>U} \frac{\left< n_r \right>}{\mathcal{N}}\\
    &= \frac{E_r-U}{\left< E_r^2 \right> -U^2} \frac{\left< n_r \right>}{\mathcal{N}}=\mathcal{RHS}
    \end{align}

\subsection*{(b-2)}
    \begin{align}
    \frac{\left< (\Delta n_r)^2 \right>}{\mathcal{N}}
    =& \omega_r \frac{\partial}{\partial \omega_r}\left[ \frac{\omega_r \exp{(-\beta E_r)}}{ \sum_r \omega_r \exp{(-\beta E_r)} } \right]\\
    =& \frac{\omega_r \exp{(-\beta E_r)}}{ \sum_r \omega_r \exp{(-\beta E_r)} }
    -\frac{\omega_r^2 E_r  \exp{(-\beta E_r)}}{ \sum_r\omega_r \exp{(-\beta E_r)} }  \frac{\partial\beta}{\partial \omega_r}\\
    &-\frac{\omega_r^2 \left(\exp{(-\beta E_r)}\right)^2 - \omega_r^2 \exp{(-\beta E_r)} \sum_r\omega_r E_r \exp{(-\beta E_r)} }
    { \left( \sum_r\omega_r \exp{(-\beta E_r)} \right)^2 }
    \frac{\partial\beta}{\partial \omega_r}\\
    =&\frac{\left< n_r \right>}{\mathcal{N}}
    -\frac{\left< n_r \right>}{\mathcal{N}}E_r \frac{\partial\beta}{\partial \omega_r}
    - \left(  \frac{\left< n_r \right>}{\mathcal{N}} \right)^2
    +\frac{\left< n_r \right>}{\mathcal{N}} U \frac{\partial\beta}{\partial \omega_r}\\
    =&\frac{\left< n_r \right>}{\mathcal{N}}
    +\frac{\left< n_r \right>}{\mathcal{N}} (U-E_r) \frac{\partial\beta}{\partial \omega_r}
    - \left(  \frac{\left< n_r \right>}{\mathcal{N}} \right)^2
    \end{align}


\section*{Problem 3.2}

    \begin{align}
    g\prime\prime(x_0)&\simeq\frac{f\prime\prime(x_0)}{f(x_0)}-\frac{U^2-U}{x_0^2}\\
    &=\frac{\sum\omega_r E_r(E_r-1)x_0^{E_r}}{x_0^2 \sum\omega_r x_0^{E_r}}-\frac{U^2-U}{x_0^2}\\
    &=\frac{\left<E_r^2\right>-\left<E_r\right>}{x_0^2} -\frac{U^2-U}{x_0^2}\\
    &=\frac{\left<E_r^2\right>-U^2}{x_0^2}\\
    &=\frac{\left(\left<E_r\right>-U\right)^2}{x_0^2}
    \end{align}


\section*{Problem 3.3}

    \begin{align}
    \exp(x)=\sum \frac{1}{n!}x^n\\
    \frac{1}{n!}=\frac{1}{2\pi\i} \oint \frac{\exp(z)}{z^{n+1}}dz
    \end{align}
    \begin{align}
    \text{Define: }
    g(z)\equiv \ln{(\frac{\exp(z)}{z^{n+1}})}\equiv \ln (F(z))
    \end{align}
    \begin{align}
    g(z)=z-(n+1)\ln z
    \end{align}
For $F(z)$, the saddle point is defined as $F\prime(x_0)=0$, which gives $x_0=n+1$. Notice that $z=x_0$ is also the saddle point for $g(z)$. Expanding $g(z)$ about the point $z=x_0$, along the line $z=x_0+iy$, we get:
    \begin{align}
    g(z)=g(x_0)-\frac{1}{2}g \prime \prime(x_0)y^2+...
    \end{align}
Thus, the integrand, along the line $z=x_0+iy$, will become:
    \begin{align}
    F(z)=\frac{\exp(x_0)}{x_0^{n+1}} \exp \left[  -\frac{1}{2}g \prime \prime(x_0)y^2  \right]
    \end{align}
    \begin{align}
    \frac{1}{n!}&=\frac{1}{2\pi\i} \oint \frac{\exp(z)}{z^{n+1}}dz\\
    &\simeq \frac{1}{2\pi\i} \frac{\exp(x_0)}{x_0^{n+1}} \int_{-\infty}^{+\infty} \exp \left[  -\frac{1}{2}g \prime \prime(x_0)y^2  \right] \i dy\\
    &=\frac{\exp(n+1)}{(n+1)^{n+1}} \frac{1}{ \left[2\pi g \prime \prime(x_0)\right]^{1/2}}\\
    &=\frac{\exp(n+1)}{(n+1)^{n+1}} \left(\frac{n+1}{2\pi}\right)^{1/2}
    \end{align}
Do a simple calculation and replace $(n+1)$ wit$n$, we get:
    \begin{align}
    n! \simeq \sqrt{2\pi n} \left( \frac{n}{e} \right)^n
    \end{align}
which is just the original form of Stirling formula for $n!$.


\section*{Problem 3.4}
    \begin{align}
    \mathcal{LHS}&=(k/\mathcal{N}) \ln\Gamma\\
    &=(k/\mathcal{N}) \ln\sum W_{n_r}\\
    &=(k/\mathcal{N}) \ln\sum \frac{\mathcal{N}!}{\Pi (n_r!)}
    \end{align}
When $\mathcal{N}$ is extremely a huge number, only the maximal set ${n_r^*}$ will make a difference. Thus:
    \begin{align}
    \sum \frac{\mathcal{N}!}{\Pi (n_r!)}&=\frac{\mathcal{N}!}{\Pi (n_r!)}\\
    &=\frac{\mathcal{N}!}{\Pi (\langle n_r \rangle !)}
    \end{align}
    \begin{align}
    \mathcal{LHS}&=(k/\mathcal{N}!) \ln\sum \frac{\mathcal{N}}{\Pi (n_r!)}\\
    &=(k/\mathcal{N}) \ln\frac{\mathcal{N}!}{\Pi (\langle n_r \rangle !)}\\
    &=(k/\mathcal{N}) \left( \mathcal{N}\ln\mathcal{N} - \sum \langle n_r \rangle \ln \langle n_r \rangle \right)\\
    &=(k/\mathcal{N}) \left( \sum \langle n_r \rangle \ln\mathcal{N} - \sum \langle n_r \rangle \ln \langle n_r \rangle \right)\\
    &=(k/\mathcal{N}) \sum \left( \langle n_r \rangle \left(\ln\mathcal{N} -\ln \langle n_r \rangle \right) \right)\\
    &=-k\sum \frac{\langle n_r \rangle}{\mathcal{N}} \ln \frac{\langle n_r \rangle}{\mathcal{N}}\\
    &=-k\langle \ln Pr \rangle\\
    &=S=\mathcal{RHS}
    \end{align}



\section*{Problem 3.5}
	Since the Helmholtz free energy $A(N,V,T)$ has the property:
	\begin{equation*}
		A(\lambda N,\lambda V,T) = \lambda A(N,V,T)
	\end{equation*}
	Differentiate with respect to $\lambda$ and substitute $\lambda=1$ immediately yields
	\begin{equation*}
		N\left( \frac{ \partial A }{\partial N} \right)_{V,T}+V \left( \frac{ \partial A }{\partial V} \right)_{N,T} = A
	\end{equation*}

\section*{Problem 3.6}

    Solving this problem is equal to calculate the most probable distribution, which we have done many times.

\section*{Problem 3.7}
\begin{eqnarray*}
C_p-C_V=(\frac{\partial H}{\partial T})_p-(\frac{\partial E}{\partial T})_V   \qquad\quad\\
=(\frac{\partial (E+pV)}{\partial T})_p-(\frac{\partial E}{\partial T})_V   \\
=p (\frac{\partial V}{\partial T})_p+(\frac{\partial E}{\partial V})_{\beta}(\frac{\partial V}{\partial T})_p  \\
=(\frac{\partial V}{\partial T})_p (p+(\frac{\partial E}{\partial V})_{\beta} ) \qquad \\
=-\frac{(\frac{\partial p}{\partial T})_V}{(\frac{\partial p}{\partial V})_T}(p-\frac{\partial^2 lnQ}{\partial \beta \partial V})  \quad \\
=-\frac{k(\beta \frac{\partial^2 ln Q}{\partial \beta \partial V}-\frac{\partial lnQ}{\partial V})^2}{(\frac{\partial^2 lnQ}{\partial V^2})_{\beta}} \quad\\
=desired \ formula.\qquad
\end{eqnarray*}
As for classical ideal gas, $(\frac{\partial E}{\partial V})_{\beta}=0$, $pV=NkT$, we soon get that the above result is Nk.


\section*{Problem 3.8}
For classical ideal gas
\begin{eqnarray*}
\ln\left(\frac{Q_1}{N}\right)+T\left(\frac{\partial \ln Q_1}{\partial T}\right)_P&=&\ln\left\{\frac{V}{h^3N}(2\pi mkT)^{3/2}\right\}+T\frac{\partial}{\partial T}\ln\left\{\frac{N}{h^3P}(2\pi m)^{3/2}(kT)^{3/5}\right\}\\
&=&\ln\left\{\frac{V}{N}(\frac{2\pi mkT}{h^2})^{3/2}\right\}+\frac{5}{2}\\
&=&\frac{S}{Nk}
\end{eqnarray*}

 \section*{Problem 3.9}
 	For an ideal monaomic gas,its heat capacity C would be 3R/2.While asume the whole progress is quasistatic,it would obey
 	\begin{equation*}
		pV=RT
	\end{equation*}
	\begin{equation*}
		dU=-pdV+dQ=CdT
	\end{equation*}
	So we can get
	\begin{equation*}
		\frac{5}{2}pdV+\frac{3}{2}Vdp=dQ
	\end{equation*}
	For adiabatical process,dQ=0,so the ratio of the final pressure to initial pressure would be
	\begin{equation*}
		\frac{p_f}{p_i}=(1/2)^{5/3}
	\end{equation*}
	For the process with heat,the equation is difficult to solve,but naively thinking,for a process that the pressure doesn't change,it need heat to be added,so the final pressure would be higher than adiabatical process.
	
	
\section*{Problem 3.11}
	Suppose $pV^{n} = C$, so the work done is
	\begin{equation}
	\Delta W = \int^{V_{2}}_{V_{1}} \frac{C}{V^{n}} dV = \frac{C}{n-1} (V^{1-n}_{2}	 - V^{1-n}_{1})
	\end{equation}
	The energy difference is given by
	\begin{equation}
	\Delta U = p_{2}V_{2} - p_{1}V_{1} = C (V^{1-n}_{2} - V^{1-n}_{1})
	\end{equation}
	Therefore, the heat absorbed is
	\begin{equation}
	\Delta Q =  C\frac{n-2}{n-1} (V^{1-n}_{2} - V^{1-n}_{1})
	\end{equation}

\section*{Problem 3.12}
	The Hamiltonian of the classical system can be written as:
	\begin{equation}
	H = \sum_{i}^{N} \frac{\mathbf{p}_i^2}{2m}+\sum_{i}^{N}U(\mathbf{x}_i)
	\end{equation}
	So the partition function of the system is:
	\begin{eqnarray}
	Q(\beta,N,V) &=& \frac{1}{N!h^{3N}}\int\prod_{i=1}^N d^3x_id^3p_i e^{-\beta H(x,p)}\nonumber\\
	&=&\frac{1}{N!}\left[\left(\frac{2\pi m\beta^{-1}}{h^2}\right)^{3N/2}\int \prod_{i}d^3x_i e^{-\beta U(\mathbf{x}_i)}\right]\nonumber\\
	\end{eqnarray}
	So the Helmholtz potential is $A = -kT\ln Q$ and the entropy $S$ is the derivative of free energy:
	\begin{eqnarray}
	S &=& -\frac{\partial A}{\partial T}\nonumber\\
	&=&-\frac{\partial}{\partial T}\left\{-kT\ln\left[\frac{1}{N!}\left(\frac{2\pi mkT}{h^2}\right)^{3N/2}\left(\int \prod_id^3x_i e^{-\beta U(\mathbf{x}_i)}\right)\right]\right\}\nonumber\\
	&=&-\frac{\partial}{\partial T}\left\{-NkT\ln\left[\frac{1}{N}\left(\frac{2\pi mkT}{h^2}\right)^{3/2}\left(\int \prod_id^3x_i e^{-\beta U(\mathbf{x}_i)}\right)^{1/N}\right]-NkT\right\}\nonumber\\
	&=&Nk\ln\left[\frac{1}{N}\left(\frac{2\pi mkT}{h^2}\right)^{3/2}\left(\int\prod_i d^3x_i e^{-\beta U(\mathbf{x}_i)}\right)^{1/N}\right]+\frac{3}{2}Nk + \frac{1}{T}\frac{\int \prod_i d^3x_i \sum_i U(\mathbf{x}_i)e^{-\beta U(\mathbf{x}_i)}}{\int \prod_i d^3x_i e^{-\beta U(\mathrm{x}_i)}} + Nk\nonumber\\
	&=&\frac{5Nk}{2} + Nk\ln\left[\frac{1}{N}\left(\frac{2\pi mkT}{h^2}\right)^{3/2}\left(\int\prod_i d^3x_i e^{-\beta U(\mathbf{x}_i)}\right)^{1/N}\right] + \frac{\overline{U}}{T}\nonumber\\
	&=&\frac{5Nk}{2} + Nk\ln\left[\frac{1}{N}\left(\frac{2\pi mkT}{h^2}\right)^{3/2}e^{\frac{\overline{U}}{NkT}}\left(\int\prod_i d^3x_i e^{-\beta U(\mathbf{x}_i)}\right)^{1/N}\right] \nonumber\\
	&=&Nk\left\{\frac{5}{2}+ \ln \left[\frac{\overline{V}}{N}\left(\frac{2\pi mkT}{h^2}\right)^{3/2}\right]\right\}
	\end{eqnarray}
	Up till now we have shown the entropy of such a system. So if the potential energy is just a constant, the ``free volume'' is the common volume of classical ideal gas.

	Then consider about the hard sphere gas. The potential energy is:
	$$
	U(\mathbf{x}_i)=\left\{\begin{array}{ll}
	0 &|\mathbf{x}_i-\mathbf{x}_j| > D\\
	\infty &|\mathbf{x}_i-\mathbf{x}_j| < D
	\end{array}\right.
	$$
	It is obvious that the average of potential energy is $\overline{U} = 0$, so the free volume is
	\begin{eqnarray}
	\overline{V}^N&=&\int \prod_i d^3x_i\,e^{-\beta U(\mathbf{x}_i)}\nonumber\\
	&=&\int d^3x_N\int d^3x_{N-1}\cdots \int d^3x_1 e^{-\beta U(\mathbf{x}_i)}\nonumber\\
	&=&V\left(V-\frac{4\pi}{3}D^3\right)\left(V-2\cdot\frac{4\pi}{3}D^3\right)\cdots\left(V-\frac{N-1}{3}4\pi D^3\right)
	\end{eqnarray}
	Define $v_0 = \pi D^3/6$ is the volume a sphere, so the gas-law will be:
	\begin{eqnarray}
	P &=& \frac{NkT}{\overline{V}}\frac{\partial \overline{V}}{\partial V}\nonumber\\
	&=& kT\left(\frac{1}{V}+\frac{1}{V-8v_0}\cdots \frac{1}{V+8(N-1)v_0}\right)\nonumber\\
	&\simeq & kT\left(\frac{N+N^2\frac{4v_0}{V}}{V}\right)\nonumber\\
	&=& kT\frac{N}{V\frac{1}{1+4Nv_0/V}}\nonumber\\
	&\simeq &\frac{NkT}{V-4Nv_0}
	\end{eqnarray}
	This result is the same as we have seen in Problem 1.4.

\section*{Problem 3.13} % (fold)
\label{sec:problem_3_13}
	(a)Use classical method,it is easy to get partition function.
	$$Q_N=\frac{1}{N_1!N_2!}[\frac{V}{h^3}(2\pi m_1kT)^\frac{3}{2}]^{N_1}[\frac{V}{h^3}(2\pi m_2kT)^\frac{3}{2}]^{N_2}$$
	For the same reason.We get the partition function of another system:
	$$Q_N=\frac{1}{(N_1+N_2)!}[\frac{V}{h^3}(2\pi mkT)^\frac{3}{2}]^{N_1+N_2}$$
	$m$ is mixed mass.
	$$m=\frac{N_1m_1+N_2m_2}{N_1+N_2}$$
% section problem_3_13 (end)


\section*{Problem 3.14}
	The Lagrangian of the system can be simply expressed as
	\begin{align}
		L&=T-V\\
		&=\sum\limits_{i}\frac{1}{2}m\dot{r_i}^2-\sum\limits_{i,j}{u(r_{ij})}-\sum\limits_{i}{u_{wall}(x_i)+u_{wall}(L-x_i)+u_{wall}(y_i)+u_{wall}(L-y_i)+u_{wall}(z_i)+u_{wall}(L-z_i)}
	\end{align}
	Applying the Legendre transform and get Hamiltonian of the system expressed as
	\begin{align}
		H=\sum\limits_{i}\frac{p_i^2}{2m}+\sum\limits_{i,j}{u(r_{ij})}+\sum\limits_{i}{u_{wall}(x_i)+u_{wall}(L-x_i)+u_{wall}(y_i)+u_{wall}(L-y_i)+u_{wall}(z_i)+u_{wall}(L-z_i)}
	\end{align}
	(a)\\
	\begin{align}
		P&=-(\frac{\partial H}{\partial V})=\frac{-1}{3L^2}(\frac{\partial H}{\partial L})\\
		&=\frac{-1}{3L^2}\sum\limits_{i}{u_{wall}'(L-x_i)+u_{wall}'(L-y_i)+u_{wall}'(L-z_i)}
	\end{align}


\section*{Problem 3.15} % (fold)
\label{sec:problem_3_15}

	We have $Q_1(V,T)=\int g(\epsilon) e^{-\beta \epsilon}d \epsilon $. For 3-D extreme relativistic gas, $\epsilon =pc $, hence we have
	\begin{gather*}
		g(p)dp=\frac{V}{h^3}4 \pi p^2dp = \frac{4 \pi V }{h^3} \frac{\epsilon^2}{c^2} \frac{d \epsilon}{c} = g(\epsilon)d \epsilon\\
		\therefore g(\epsilon)=\frac{4 \pi V}{(hc)^3} \epsilon^2\\
		\therefore Q_1(V,T)=\int^{\infty}_0 g(\epsilon)d \epsilon=\frac{4 \pi V}{(hc)^3} \int^{\infty}_0 \epsilon^2 e^{-\beta \epsilon}d \epsilon = 8 \pi V \left( \frac{kT}{hc} \right)^3
	\end{gather*}
	
	$\therefore$for $N$ molecules,
	\begin{equation*}
		Q_N(V,T)= \frac{1}{N!}\left\{ 8 \pi V \left( \frac{kT}{hc} \right)^3\right\}^N
	\end{equation*}

	From $Q_N(V,T)$, it's easy to calculate:
	\begin{gather*}
		P=\frac{1}{\beta} \frac{\partial Q}{\partial V} = \frac{N}{V}kT\\
		U = -\frac{1}{Q} \frac{\partial Q}{\partial \beta}=3NkT\\
		\gamma = \frac{4}{3}
	\end{gather*}


	As stated in section 3.4, $g(E) $ can be obtained from the inverse Laplace transform, i.e.,
	\begin{equation*}
		g(E)=\frac{1}{2 \pi i}\int_{\beta'-i\infty}^{\beta'+i\infty} e^{\beta E} Q(\beta)d \beta
	\end{equation*}
	in our case, $Q(\beta)=Q_N(V, T) $, hence
	\begin{eqnarray*}
		g(E) &=& \frac{1}{2 \pi i}\int_{\beta'-i\infty}^{\beta'+i\infty} e^{\beta E} Q(\beta)d \beta\\
		&=& \frac{(8 \pi  V)^N}{N!(hc)^{3N} } \text{Res}\left[ \frac{e^{\beta E}}{\beta^{3N}} \right]_{\beta=0}\\
		&=& \frac{(8 \pi  V)^N E^{3N-1}}{N!(3N-1)!(hc)^{3N} }
	\end{eqnarray*}


% section problem_3_15 (end)

\section*{Problem 3.16}

    We can get the partition function of the system by utilizing equation $(3.5.5)$:
    \begin{equation}
        Q_N(V,T)=\frac{1}{N! h^{3N}}\int e^{-\beta H(q,p)}d\omega
        \label{JC_Li_001}
    \end{equation}
    Since the particles in this system obey the energy-momentum relationship $\epsilon=pc$, and the particles can only move in one dimension,  equation.\eqref{JC_Li_001} becomes:
    \begin{equation}
        Q_N(L,T)=\frac{1}{(3N)! h^{3N}}\int e^{-\beta |p|c}d^{3N}p d^{3N}x
    \end{equation}
    Then we can get the partition function:
    \begin{equation}
        Q_N(V,T)=\frac{1}{(3N)!}\left[2L\frac{kT}{hc}\right]^{3N}
        \label{JC_Li_002}
    \end{equation}
    And we can study the thermodynamics of this system:
    $$
        P=\frac{1}{\beta} \frac{\partial \ln Q}{\partial L} = \frac{3 N}{V}kT
    $$
    $$
        U = - \frac{\partial \ln Q}{\partial \beta}=3NkT
    $$
    $$
        \gamma = \frac{4}{3}
    $$

    Using the inversion formula $(3.4.7)$, we can derive an expression of the density of states $g(E)$. From equation $(3.4.7)$:
    \begin{equation}
    \begin{aligned}
        g(E)&=\frac{1}{2\pi i}\int_{\beta'-i\infty}^{\beta'+i\infty}e^{\beta E}Q(\beta)d\beta\\
            &=\frac{1}{2\pi i}\int_{\beta'-i\infty}^{\beta'+i\infty}\frac{1}{(3N)!}\left[\frac{2L}{hc}\right]^{3N}\frac{e^{\beta E}}{\beta^{3N}}d\beta
    \end{aligned}
        \label{JC_Li_003}
    \end{equation}
    Since the integrand have only one singularity, $\beta=0$, we can calculate this integration by using Residue Theorem:
    \begin{equation}
    \begin{aligned}
        g(E)&=\frac{1}{2\pi i}\frac{E^{3N-1}}{(3N)!}\left[\frac{2L}{hc}\right]^{3N}\int_{\beta'-i\infty}^{\beta'+i\infty}\frac{e^{\beta E}}{(\beta E)^{3N}}d(\beta E)\\
            &=\frac{1}{2\pi i}\frac{E^{3N-1}}{(3N)!}\left[\frac{2L}{hc}\right]^{3N}\int_{\beta'-i\infty}^{\beta'+i\infty}\frac{1}{(\beta E)^{3N}}\sum_{j=0}^{\infty}\frac{E^j}{j!} d(\beta E)\\
            &=\frac{1}{2\pi i}\frac{E^{3N-1}}{(3N)!(3N-1)!}\left[\frac{2L}{hc}\right]^{3N}
    \end{aligned}
        \label{JC_Li_004}
    \end{equation}


\section*{Problem 3.17}
\begin{eqnarray*}
\int [U-H(p,q)]e^{-\beta H(p,q)}d\omega=0 \\
\Rightarrow \int [\frac{\partial U}{\partial \beta}-H(p,q)U+H^2(p,q)]e^{-\beta H(p,q)}d\omega=0 \\
\Rightarrow \int [\frac{\partial U}{\partial \beta}-U^2+H^2(p,q)]e^{-\beta H(p,q)}d\omega=0  \\
\Rightarrow \langle H^2\rangle-U^2=-\frac{\partial U}{\partial \beta}
\end{eqnarray*}
That is the desired equation.
\section*{Problem 3.18}
\begin{eqnarray*}
\langle(\Delta E)^3\rangle&=&\langle E^3-2E^2\langle E\rangle +2E\langle E\rangle ^2-\langle E\rangle ^3\rangle \\&=&\langle E^3\rangle -2\langle E^2\rangle \langle E\rangle +\langle E\rangle^3
\end{eqnarray*}
Considering the relations below
\begin{eqnarray*}
\langle E\rangle &=&\frac{E_r\mathrm{exp}(-\beta E_r)}{\mathrm{exp}(-\beta E_r)}\\
\langle E^2\rangle &=&\frac{E_r^2\mathrm{exp}(-\beta E_r)}{\mathrm{exp}(-\beta E_r)}
\end{eqnarray*}
\begin{equation*}
C_V=\frac{\langle E^2\rangle-\langle E\rangle^2}{kT^2}
\end{equation*}
\begin{equation*}
k^2\left\{T^4\left(\frac{\partial C_V}{\partial T}\right)_V+2T^3C_V\right\}=-\frac{1}{\beta^2}\frac{\partial}{\partial \beta}\left\{\beta^2(\langle E^2\rangle-\langle E\rangle^2)\right\}+\frac{2}{\beta}(\langle E^2\rangle-\langle E^2\rangle)\\
\end{equation*}
We have
\begin{equation*}
\langle E^3\rangle=k^2\left\{T^4\left(\frac{\partial C_V}{\partial T}\right)_V+2T^3C_V\right\}
\end{equation*}


\section*{Problem 3.19}
	\begin{equation*}
		<\frac{dG}{dt}>=<\sum p_i \frac{dq_i}{dt}>+<\sum q_i \frac{dp_i}{dt}>=0
	\end{equation*}
	Above equation has used equation (3.7.5) and equation (3.7.6).
	The equation (3.7.5) and equation(3.7.6)  both come from (3.7.2),so validity of one equation implies another's.
	

\section*{Problem 3.21}
	\noindent (a) Classically, the harmonic equation of motion leads to $x = A \sin \omega t$. As a result, the kinetic energy and potential energy will be $m \omega^{2} A^{2} \cos^{2} \omega t /2$ and $m \omega^{2} A^{2} \sin^{2} \omega t /2$ respectively. Average them it's easy to see that $\bar{K} = \bar{U} =m \omega^{2} A^{2}/4$.\\
	Quantum-mechanically, $\psi = \sum_{n} c_{n} \psi_{n}$ where $\psi_{n}$ is the \emph{n}-th Hermitian polynomial. Using the recursive relations, we have
	\begin{align}
	&\bar{K} = -\frac{\hbar^{2}}{2m} \sum_{n} |c_{n}|^{2} \int \psi^{*} \frac{d^{2}}{dx^{2}} \psi dx = \sum_{n} |c_{n}|^{2} \frac{\hbar \omega (2n+1)}{4} = \frac{1}{2} \sum_{n} |c_{n}|^{2} E_{n}\\
	&\bar{U} = \frac{m \omega^{2}}{2} \sum_{n} |c_{n}|^{2} \int \psi^{*} x^{2} \psi dx= \sum_{n} |c_{n}|^{2} \frac{\hbar \omega (2n+1)}{4} = \frac{1}{2} \sum_{n} |c_{n}|^{2} E_{n}
	\end{align}
	\noindent (b) In Bohr-sommerfeld model, a quantized orbits are hypothesized, namely $m_{e}vr = n \hbar$. In the \emph{n}-th orbit, the total energy is $E_{n} = - Z^{2}k^{2}e^{4}m_{e}/2 \hbar^{2} n^{2}$. The radius of which is $r_{n} = n^{2} \hbar^{2} / Zke^{2}m_{e}$. By a naive calculation $\bar{U} = - Z^{2}k^{2}e^{4}m_{e}/ \hbar^{2} n^{2}$ and $\bar{T} = Z^{2}k^{2}e^{4}m_{e}/2 \hbar^{2} n^{2}$.\\
	In the Schroedinger hydrogen atom, $\psi_{nlm} = R_{nl}(r) Y_{lm}(\theta, \phi)$. The kinetic energy is given by
	\begin{eqnarray}
	\bar{T} &=& \frac{\hbar^{2}}{2m} \int \psi^{*}_{nlm} (\frac{d^{2}}{dr^{2}} + \frac{2}{r} \frac{d}{dr} - \frac{l(l+1)}{r^{2}})\psi_{nlm} r^{2} \sin \theta dr d\theta d\phi \notag \\
	&=& \frac{\hbar^{2}}{2m} \int R_{nl}(r) (\frac{1}{n^{2}a^{2}}) R_{nl}(r) r^{2}dr \notag \\
	&=& \frac{e^{2}}{2a n^{2}}
	\end{eqnarray}
	so $\bar{U} = - e^{2}/a n^{2}$. \emph{a} is the Bohr radius.\\
	\noindent (c) This is also a central force case. The results are quite identical to (b).

\section*{Problem 3.22}
	Anharmonic Oscillator.

	This anharmonic oscillator has the Hamiltonian:
	$$
	H = \frac{p^2}{2m} + \frac{1}{4}kx^4
	$$
	So the canonical partition function of the system is:
	\begin{equation}
	Q = \frac{1}{h}\int dpdx\,e^{-\beta\left(\frac{p^2}{2m} + \frac{1}{4}kx^4\right) }
	\end{equation}
	Use the ``equipartition theorem'', we can get the following result:
	\begin{equation}
	\Big{\langle} x \frac{\partial H}{\partial x}\Big{\rangle} =kT
	\end{equation}
	Thus because $\partial H/\partial x = kx^3$, we can get
	$$
	x\frac{\partial H}{\partial x} = kx^4 = 4V
	$$
	So the expectation value of the potential is $\langle V\rangle = kT/4$. For the same reason, we can get the mean value of the kinetic energy:
	\begin{equation}
	\langle K\rangle = \frac{1}{2}\Big{\langle} p\frac{\partial H}{\partial p}\Big{\rangle} = \frac{kT}{2}
	\end{equation}
	So clearly we can get $\langle K\rangle = 2\langle V\rangle$.
	

\section*{Problem 3.23} % (fold)
\label{sec:problem_3_23}
	According to the equation 3.7.15
	$$
	\frac{PV}{NkT}=1-\frac{1}{NdkT}*\overline{\sum_{i<j}\frac{\partial{u(r_{ij})}}{\partial{r_{ij}}}r_{ij}}
	$$
	For the ideal gas.There is not interaction term.
	$$
	PV=NkT
	$$
	The Hamiltonian of the system happens to be a quadratic function of its coordinates.The virial theorem states that
	$$
	\boldsymbol{\nu_0}=-3NkT
	$$
	So we can infer that
	$$
	\boldsymbol{\nu_0}=-3PV
	$$
	Let's consider the interaction between the particles and walls of container.
	$$
	\boldsymbol{\nu_0}=-P\mathlarger{\int}(\nabla\cdot\boldsymbol{ r})dV=-3PV
	$$
	They show walls of container are the main factor interaction with particles.
% section problem_3_23 (end)

\section*{Problem 3.24}
	The relativistic dispersive relation of free particle can be expressed as
	\begin{align}
		\left\langle p\cdot u \right\rangle &=3kT\\
		3kT&=\left\langle p\cdot u \right\rangle =\left\langle\gamma m_0u \cdot u \right\rangle =\left\langle\frac{m_0u^2}{\sqrt{1-\frac{u^2}{c^2}}} \right\rangle 
	\end{align}
	In the extreme relativistic case, the thermal energy per particle can be expressed as
	\begin{align}
		u&\rightarrow c\\
		\left\langle E \right\rangle & \approx \left\langle pc \right\rangle \approx \left\langle pu\right\rangle=3kT
	\end{align}
	While in non-relativistic case
	\begin{align}
		\left\langle E\right\rangle=\left\langle\frac{1}{2}mu^2\right\rangle=\frac{3}{2}kT
	\end{align}


\section*{Problem 3.25} % (fold)
\label{sec:problem_3_25}
	Consider a particle inside a box with $\dot{q_i} $ and $p_i$, the volumn of the box is V. If the particle hits an area $\Delta S $ on a wall during time $\Delta t $, it has to be in volumn $\dot{q_i} \Delta S \Delta t $. Also, the momentum $p_i$ it has must be oriented to the wall, which gives a $1/2$ coefficient to the probability. Hence the pressure on the wall satisfies
	\begin{gather*}
		\left<\sum_N \frac{1}{2} \frac{\dot{q_i} \Delta S \Delta t }{V} \cdot 2p_i \right> = P \Delta S \Delta t\\
		i.e., \left<\sum_i p_i \dot{q_i} \right> = 3PV, i=1,...,3N
	\end{gather*}

	From the equipartition theorem, $\left<\sum_i p_i \dot{q_i} \right> = 3NkT  $, hence
	\begin{equation*}
		PV=NkT
	\end{equation*}
	for noninteracting systems.
% section problem_3_25 (end)

\section*{Problem 3.26}

    To calculate the multiplicity of an s-dimensional oscillator, we can write the energy eigenvalues in this form:
    \begin{equation}
        \epsilon_j=(j+s/2)\hbar \omega=(\sum_{i=1}^s n_i+s/2)\hbar \omega
    \end{equation}
    Where $n_i$ is the ``eigenvalues'' of each dimension. And $n_i$ can be a integer between $0$ to $j$, just have to obey $\sum_{i=1}^s n_i=j$. So this problem is equivalent to putting $s-1$ ``clapboards'' between $N$ particles. Hence we can get the multiplicity:
    \begin{equation}
        m_j=\frac{(j+s-1)!}{j!(s-1)!}
    \end{equation}
    Then we can get the partition function of a single oscillator:
    \begin{equation}
    \begin{aligned}
        Q_1&=\sum_j m_j \exp\left(-\beta \epsilon_j\right)\\
            &=\frac{(j+s-1)!}{j!(s-1)!}\exp\left[-\beta (j+s/2)\hbar \omega\right]\\
            &=\left[\frac{\exp (-\beta\hbar \omega/2)}{1-\exp (-\beta \hbar \omega)}\right]^s
    \end{aligned}
    \end{equation}
    The partition function of a system of N oscillators is:
    \begin{equation}
        Q_N=Q_1^N=\left[\frac{\exp (-\beta\hbar \omega/2)}{1-\exp (-\beta \hbar \omega)}\right]^{sN}
        \label{JC_Li_005}
    \end{equation}
    We can study the thermodynamics of this system from equation.\eqref{JC_Li_005}:
    And we can study the thermodynamics of this system:
    $$
        U = - \frac{\partial \ln Q_N}{\partial \beta}= sN\left[\frac{1}{2}+\frac{1}{1-\exp(-\beta \hbar \omega)}\right]\hbar\omega
    $$
    $$
        \mu_s = - \frac{1}{\beta}\frac{\partial \ln Q_N}{\partial N}= s\left[\frac{\hbar\omega}{2}+\frac{\ln [1-\exp(-\beta \hbar \omega)]}{\beta}\right]
    $$
    For a system of sN one-dimensional oscillators:
    $$
        Q_{sN}=Q_1^sN=\left[\frac{\exp (-\beta\hbar \omega/2)}{1-\exp (-\beta \hbar \omega)}\right]^{sN}
    $$
    $$
        U = - \frac{\partial \ln Q_N}{\partial \beta}= sN\left[\frac{1}{2}+\frac{1}{1-\exp(-\beta \hbar \omega)}\right]\hbar\omega
    $$
    $$
        \mu_1 = - \frac{1}{\beta}\frac{\partial \ln Q_N}{\partial sN}= \left[\frac{\hbar\omega}{2}+\frac{\ln [1-\exp(-\beta \hbar \omega)]}{\beta}\right]
    $$
    And we have:
    $$
        \mu_s=s\mu_1
    $$


\section*{Problem 3.27}
\begin{eqnarray*}
g(E)=\frac{1}{2 \pi}\int_{-\infty}^{+\infty} e^{(\beta +i\gamma)(E-(N/2)\hbar\omega)}(1-e^{-\hbar\omega(\beta +i\gamma)})^{-N}d\gamma  \qquad \qquad \qquad \qquad \qquad \qquad \qquad \qquad \qquad \quad\\
=\frac{1}{2 \pi}\int_{-\infty}^{+\infty} e^{(\beta +i\gamma)(E-(N/2)\hbar\omega)}(1+C_N^1e^{-\hbar\omega(\beta +i\gamma)}+C_{N+1}^2e^{-2\hbar\omega(\beta +i\gamma)}+C_{N+2}^3e^{-3\hbar\omega(\beta +i\gamma)}+...)d\gamma  \\
=\sum_{k=0}^{\infty}\delta(E-(N/2)\hbar\omega-k\hbar\omega)C_{N+k-1}^{k}  \qquad \qquad \qquad \qquad \qquad \qquad \qquad \qquad \qquad \qquad \qquad \qquad \qquad
\end{eqnarray*}
The above result is the same as that derived from direct state counting for N distinguishable quantum SHOs. Assume $ E=(m+N/2)\hbar\omega $, then we have:
\begin{eqnarray*}
S=k\ln(g(E)dE)=k\ln(C_{N+m-1}^m)
\end{eqnarray*}
while $m\gg1$, $N\gg1$, using the Stiring formula we can find:
\begin{eqnarray*}
S\approx k(m\ln{\frac{N+m-1}{m}}+(N-1)\ln{\frac{N+m-1}{N-1}})\approx Nk(\frac{m+N}{N}\ln{\frac{m+N}{N}}-\frac{m}{N}\ln{\frac{m}{N}})
\end{eqnarray*}
which is the desired formula.
\section*{Problem 3.28}
\subsection*{a)}
Define
\begin{equation*}
R=\left(E-\frac{1}{2}N\hbar\omega\right)\Bigg/\hbar\omega
\end{equation*}
Number of states avaliable for the whole system is
\begin{equation*}
m_0=\frac{(R+N-1)!}{R!(N-1)!}
\end{equation*}
Number of states avaliable for a particular oscillator in state n
\begin{equation*}
m=\frac{(R+N-1-n-1)!}{(R-n)!(N-1)!}
\end{equation*}
Probability
\begin{equation*}
p_n=\frac{m}{m_0}=\frac{R(R-1)\cdots(R-n+1)(N-1)}{(R+N-1)\cdots(R+N-1-n-1)}
\end{equation*}
for $N\gg 1$ and $R\gg n$
\begin{equation*}
p_n\approx \frac{(\bar n)^n}{(\bar n+1)^{n+1}}
\end{equation*}
where $\bar n=R/N$
\subsection*{b)}
The number of states avaliable for total energy E and N particles are
\begin{equation*}
g(E,N)=\frac{1}{N!}(\frac{V}{h^3})^N\frac{(2\pi m )^{3N/2}}{(3N/2-1)!}E^{3N/2-1}
\end{equation*}
Probability
\begin{equation*}
p=\frac{g(E-\epsilon,N-1)}{g(E,N)}
\end{equation*}
For $N\ll1$ and $E\ll\epsilon$
\begin{equation*}
p\propto\left(\frac{E-\epsilon}{E}\right)^{3N/2}\approx \mathrm{exp}(-\beta \epsilon)
\end{equation*}
where $\beta=3N/2E$.

\section*{Problem 3.29}
	I can't solve this problem.The intergral of the unharnomic terms in the partition fuction is infinite.
	
\section*{Problem 3.31}
	``Partition function'' for single particle is
	\begin{equation}
	Q_{1} = 1 + e^{-\varepsilon/kT}.
	\end{equation}
	So a list of quatities can be obtained:
	\begin{align}
	&Q_{N} = (1 + e^{-\varepsilon/kT})^{N} \\
	&A = - NkT \ln (1 + e^{-\varepsilon/kT}) \\
	&\mu = - kT \ln (1 + e^{-\varepsilon/kT}) \\
	&p=0 \\
	&S = Nk \ln (1 + e^{-\varepsilon/kT}) + \frac{N \varepsilon}{T} \frac{e^{-\varepsilon/kT}}{1+e^{-\varepsilon/kT}} \\
	&U =  N \varepsilon \frac{e^{-\varepsilon/kT}}{1+e^{-\varepsilon/kT}} \\
	&C_{p} = C_{V} =  \frac{N \varepsilon^{2} e^{-\varepsilon/kT}}{kT^{2} (1+e^{-\varepsilon/kT})^{2}}
	\end{align}
	This specific heat is sometimes referred to \emph{Schottky anomaly}.

\section*{Problem 3.32}
	\paragraph{(a)} Since the distribution is given by canonical distribution, the probabilities are:
	$$
	p_i = Q^{-1}g_ie^{-\beta \epsilon_i}
	$$
	and the entropy should be:
	\begin{eqnarray}\label{entropy}
	S &=& -k\left[p_1\ln(p_1/g_1)+p_2\ln(p_2/g_2)\right]\nonumber\\
	&=& -k\left[\frac{g_1e^{-\beta \epsilon_1}}{Q}\ln \frac{e^{-\beta\epsilon_1}}{Q}+\frac{g_2e^{-\beta \epsilon_2}}{Q}\ln \frac{e^{-\beta\epsilon_2}}{Q}\right]\nonumber\\
	&=& k\ln Q + \frac{1}{T}\frac{g_1\epsilon_1e^{-\beta\epsilon_1}+g_2\epsilon_2e^{-\beta\epsilon_2}}{Q}\nonumber\\
	&=&k\ln g_1 + k\ln\left(1+\frac{g_2}{g_1}e^{-x}\right)+\frac{1}{T}\frac{g_2(\epsilon_2-\epsilon_1)e^{-\beta\epsilon_2}}{Q}\nonumber\\
	&=&k\left[\ln g_1 + \ln\left(1+\frac{g_2}{g_1}e^{-x}\right)+\frac{g_2e^{-\beta\epsilon_2}x}{Q}\right]\nonumber\\
	&=&k\left[\ln g_1 + \ln\left(1+\frac{g_2}{g_1}e^{-x}\right)+\frac{x}{1+\frac{g_1}{g_2}e^x}\right]
	\end{eqnarray}
	When $g_1=g_2=1$, the situation is the same as Fermi oscillator with energy $0$ and $\epsilon_2-\epsilon_1$.
	\paragraph{(b)}
	The entropy is the derivative of the free energy, so we can get the entropy by the following process:
	\begin{eqnarray}
	S &=& -\frac{\partial A}{\partial T}\nonumber\\
	&=&\frac{\partial}{\partial T}\left\{kT\ln Q\right\}\nonumber\\
	&=&k\ln Q +\frac{1}{T}\frac{g_1\epsilon_1e^{-\beta\epsilon_1}+g_2\epsilon_2e^{-\beta\epsilon_2}}{Q}\nonumber\\
	&=&k\left[\ln g_1 + \ln\left(1+\frac{g_2}{g_1}e^{-x}\right)+\frac{x}{1+\frac{g_1}{g_2}e^x}\right]
	\end{eqnarray}
	which is the same as we get in (a).
	\paragraph{(c)}
	Clearly from equation (\ref{entropy}), when temperature is $T=0$, the entropy will be:
	\begin{equation}
	S = \lim_{x\rightarrow +\infty}k\left[\ln g_1 + \ln\left(1+\frac{g_2}{g_1}e^{-x}\right)+\frac{x}{1+\frac{g_1}{g_2}e^x}\right]=k\ln g_1
	\end{equation}
	From the distribution of canonical ensemble, we know that when the temperature is $T=0$, the system will stay on the ground state. Since the ground is g-fold degenerate, there are $g_1$ possible states. So the entropy is $S = k\ln g_1$.


\section*{Problem 3.33} % (fold)
\label{sec:problem_3_33}
	Let's consider parameter $\frac{\mu H}{kT}$.\\
	If you plot the Langevin's function:
	$$L(x)=coth(x)-\frac{1}{x}$$
	You will find when $\frac{\mu H}{kT}=5.12$ magnetic moment is saturated.\\
% section problem_3_33 (end)

\section*{Problem 3.34}
	\begin{align}
		M_z&=N\mu_z=N\mu(coth(\beta \mu H)-\frac{1}{\beta \mu H})\\
		&\approx N\mu (\frac{\beta \mu H}{3}-\frac{(\beta \mu H)^3}{45}+\cdots)\\
		&\approx \frac{N\mu^2}{3kT}H
	\end{align}
	\begin{align}
		\frac{N\mu^2}{3kT}=1.80\times10^{-6} mks
	\end{align}
	\begin{align}
		pV=NkT\\
		N=pV/kT\\
		\mu=\sqrt{\frac{1.8\times10^{-6}mks \times3k^2T^2}{pV}}
	\end{align}

\section*{Problem 3.35} % (fold)
\label{sec:problem_3_35}
	For $ \epsilon=\frac{p^2}{2m}+\left\{ \frac{p_{\theta}^2}{2I}+\frac{p_{\phi}}{2I\sin^2 \theta} \right\} - \mu E \cos \theta $, just calculate
	\begin{eqnarray*}
		Q &=&  \frac{1}{h^3} \int e^{-\beta \epsilon} d^3pd^3q\\
		&=&\int^{\infty}_0 \exp\left(- \frac{\beta p^2}{2m} \right)dp\int^{\infty}_0 \exp\left(- \frac{\beta p_{\theta}^2}{2I} \right)dp_{\theta}\int^{\infty}_0 \exp\left(- \frac{\beta p_{\phi}^2}{2I\sin^2 \theta} \right)dp_{\phi}\int\exp(- \mu E \cos \theta) drd \theta d \phi\\
		&=&\frac{2 \pi I}{\beta} \sqrt{ \frac{2 \pi m}{\beta} } \int^R_0 dr \int^{\pi}_0 \sin \theta \exp(-\mu E \cos\theta)d \theta \int^{2 \pi}_0 d \phi\\
		&=&\frac{4 \pi^2 IR}{\beta} \sqrt{ \frac{2 \pi m}{\beta} }\frac{e^{\mu E}-e^{-\mu E}}{\mu E}
	\end{eqnarray*}

	\begin{equation*}
		\therefore Q_N = \frac{1}{N!} Q^N
	\end{equation*}

	Once $Q_N$ is obtained, all thermodynamics of the system can be obtained. I forget the definition of electric polarization, etc. I hope you can obtain them from $Q_N$ by yourself.
% section problem_3_35 (end)

\section*{Problem 3.36}

    The potential energy between the two dipoles can be shown as $\epsilon(R,\theta,\varphi)$. So the force between the two dipoles can be expressed as:
    \begin{equation}
        \bm{F}=\nabla \varepsilon
    \end{equation}
    As the orientations are governed by a canonical distribution. We can write the expression of the average force:
    \begin{equation}
        \langle\bm{F}\rangle=\int\nabla \varepsilon\exp \left[ -\beta \epsilon \right] \rho(\theta,\varphi)d\theta d\varphi
    \end{equation}
    According to symmetry, we can know that the force is oriented in the connection between of the two dipoles, so the average force have the expression:
    \begin{equation}
    \begin{aligned}
        \langle F\rangle&=\int \frac{\partial \varepsilon}{\partial R}\exp \left[ -\beta \epsilon \right]\rho(\theta,\varphi)d\theta d\varphi d\theta' d\varphi'\\
            &=3 A\int \frac{\mu\mu'}{R^4}\left[ 2\cos \theta \cos \theta'-\sin\theta\sin\theta'\cos (\varphi-\varphi') \right]\exp \left[ -\beta \frac{\mu\mu'}{R^3}\left[ 2\cos \theta \cos \theta'-\sin\theta\sin\theta'\cos (\varphi-\varphi')  \right] \right]\sin\theta \sin\theta'd\theta d\varphi d\theta' d\varphi'
    \end{aligned}
    \end{equation}
    Where $A=1/\int \exp \left[ -\beta \frac{\mu\mu'}{R^3}\left[ 2\cos \theta \cos \theta'-\sin\theta\sin\theta'\cos (\varphi-\varphi')  \right] \right]\sin\theta \sin\theta'd\theta d\varphi d\theta' d\varphi'$. At high temperatures, A equals to $(4\pi)^2$. And we can expand the expression of $\langle F\rangle $ and easily calculate the integral:
    \begin{equation}
        \langle \bm{F}\rangle=-\frac{2\mu \mu'}{kT}\frac{\hat{\bm{R}}}{R^7}
    \end{equation}    
\section*{Problem 3.37}
Prof.Ni's ppt has given a detailed and complete solution to this problem. If someone regards it worthwhile I will renew this text later.
\section*{Problem 3.38}
As defined in the problem, we examine the patition function
\begin{eqnarray*}
Q_1(\beta)&=&\int_{-J}^J\mathrm{exp}(\beta g \mu_BmH)\\
&=&\frac{1}{\beta g \mu_BH}(\mathrm{exp}(\beta g \mu_BJH)-\mathrm{exp}(\beta g \mu_B JH))
\end{eqnarray*}
Choose $x=\beta g \mu_BJH$\\
Thermal dynamic properties
\begin{eqnarray*}
\bar{\mu}_z&=&\frac{1}{\beta}\frac{\partial}{\partial H}\ln Q_1(\beta)\\
&=&J^2g\mu_B(\coth (x)-\frac{1}{x})
\end{eqnarray*}

\section*{Problem 3.39}
	By using equation (3.9.18) we could get
	\begin{equation*}
		Q=\sum_{m=-1/2}^{1/2}\exp(\beta g \mu_b mH)=\exp(-\beta g \mu_b H/2)(1+\exp(\beta g \mu_b H))
	\end{equation*}
	The mean magnetic moment is
	\begin{equation*}
		M=\frac{N}{\beta}\frac{\partial}{\partial H}\ln Q=\frac{1}{2}N\mu_b g\frac{1-\exp(-\beta g \mu_b H)}{1+\exp(-\beta g \mu_b H)}
	\end{equation*}
	While the number of parallel atoms $N_+$ and antiparallel $N_-$ satisfied that
	\begin{equation}
		\left\{
		\begin{aligned}
		\overset{.} N_+ +N_- =N\\
		(N_+ -N_-) g \mu_b J =M\\
		J=1/2
		\end{aligned}
		\right.
	\end{equation}
	So we could get the answer
	\begin{equation}
		\left\{
		\begin{aligned}
		\overset{.}
		{N_+}/{N}=\frac{1}{1+\exp(-\beta g \mu_b H)}\\
		{N_-}/{N} =\frac{\exp(-\beta g \mu_b H)}{1+\exp(-\beta g \mu_b H)}\\
		\end{aligned}
		\right.
	\end{equation}
	Acoording to the given situation,flux density 0.1 $weber/m^2$ and temperature of 1000K,the respective fractions are
	\begin{equation}
		\left\{
		\begin{aligned}
		\overset{.}
		{N_+}/{N}=50.00168\%\\
		{N_-}/{N} =49.99832\%\\
		\end{aligned}
		\right.
	\end{equation}
	
\section*{Problem 3.41}
	The equilibrium temperature will be positive, since the energy of the whole system is not bounded from above. This case is a bit like the spin and lattice case. For the subsystem of spins, its energy is bounded from above, so it is possible to attain a negative temperature. While the subsystem of lattice, i.e. ideal gas in this problem, only has positive temperature. The whole system doesn't have a energy limit, so the temperature will only be positive. And energy may flow from the spin subsystem to the ideal gas.

\section*{Problem 3.42}
	Paramagnetic system.

	For a given energy $E$, we can know that:
	\begin{eqnarray}
	E &=& \mu_B H (N_{\uparrow}-N_\downarrow)\\
	N &=& N_\uparrow + N_\downarrow
	\end{eqnarray}
	So the occupying number of up(down)-spin is
	$$
	N_\uparrow = \frac{1}{2}\left(N+\frac{E}{\mu_B H}\right)\quad N_\downarrow = \frac{1}{2}\left(N-\frac{E}{\mu_B H}\right)
	$$
	And the number of the possible states will be:
	\begin{equation}
	\Omega(N,E) = \mathrm{C}_{N}^{N_\uparrow}=\frac{N!}{N_\uparrow !N_\downarrow!}
	\end{equation}
	So the entropy in micro canonical ensemble representation is:
	\begin{eqnarray}
	S &=& k\ln\Omega(E,N) \nonumber\\
	&=& Nk\ln N -N_\uparrow k \ln N_\uparrow - N_\downarrow k \ln N_\downarrow\nonumber\\
	&=& Nk\ln N -k\frac{N\mu_B H +E}{2\mu_B H}\ln \frac{N\mu_B H +E}{2\mu_B H}-k\frac{N\mu_B H -E}{2\mu_B H}\ln\frac{N\mu_B H -E}{2\mu_B H}
	\end{eqnarray}
	This result is the same as (3.10.9) in Pathria's Book. Then the temperature:
	\begin{eqnarray}
	\frac{1}{T} &=& \frac{\partial S}{\partial E}\nonumber\\
	&=&-\frac{k}{2\mu_B H}\ln \frac{N\mu_B H +E}{2\mu_B H}-\frac{k}{2\mu_B H}+\frac{k}{2\mu_BH}\ln\frac{N\mu_B H-E}{2
	mu_B H}+\frac{k}{2\mu_BH}\nonumber\\
	&=&\frac{k}{2\mu_BH}\ln\left(\frac{N\mu_BH-E}{N\mu_BH+E}\right)
	\end{eqnarray}
	And this result is also the same as equation (3.10.8).


\section*{Problem 3.43} % (fold)
\label{sec:problem_3_43}
	The hamiltonian of the system is :
	$$\boldsymbol{H}=e\phi (\boldsymbol{q})+\frac{1}{2m_e}\sum_{i=1}^{N}(\boldsymbol{P_i}-\frac{e}{c}\boldsymbol{A_i})^2$$
	$$\dot{q_i}=-\frac{\partial{H}}{\partial{p_i}}\propto p_i$$
	On the other hand
	$$\vec\mu=\frac{e}{2c}\vec{r}\times\vec{v}=\sum_{i=1}^{N}\vec{a_i}\cdot \dot{q_i}$$
	$a_i$ are vector coefficients depending on the position coordinates.
	$$\overline\mu=\frac{\mathlarger{\int}\mu*d\omega}{\mathlarger{\int}d\omega}\propto\mathlarger{\int_{ - \infty }^{ + \infty }}p*dp=0$$
	The integrand is an odd function of $p$, so it vanishes.
% section problem_3_43 (end)

<<<<<<< Updated upstream



\section*{Problem 3.44}
	(a)\\
	Assume there are $N$ distinguishable messages. The probabilities of these messages are $p_i$, where $i$ ranges from 1 to N. Due to the probability property of$p_i$, they must satisfy $\sum\limits_{i=1}^{N}{p_i}=1$. In order to determine the maximum of the Shannon entropy, we should intruduce langrange multiplier $\lambda$ to the expression.\\
	\begin{align}
		L&=I+\lambda((\sum\limits_{i=1}^{N}{p_i})-1)\\
		&=-\sum\limits_{i=1}^{N}{p_i \ln(p_i)}+\lambda((\sum\limits_{i=1}^{N}{p_i})-1)\\
	\end{align}
	The maximum is reached when
	\begin{align}
		&\frac{\partial L}{\partial p_i}=0\\
		&\frac{\partial L}{\partial \lambda}=0
	\end{align}
	This can be evaluated to be
	\begin{align}
		&-1-\ln(p_i)+\lambda=0\\
		&\sum\limits_{i=1}^{N}{p_i}=1\\
		&\rightarrow Ne^{\lambda-1}=1\\
		&\rightarrow \lambda=1+\ln(\frac{1}{N})\\
		&\rightarrow p_i=e^{\lambda-1}=\frac{1}{N}
	\end{align}
	It is easy to show that this is the maxima of the entropy.\\
	(b)\\
	For uncorrelated character set of size $N$ in which the probability of every character is $p_i$
	\begin{align}
		I_1&=\sum\limits_{i=1}^{N}{-p_i\ln(p_i)} \\
		I_2&=\sum\limits_{i,j=1}^{N,N}{-p_ip_j\ln(p_ip_j)}\\
		&=\sum\limits_{i,j=1}^{N,N}{-p_ip_j(\ln(p_i)+\ln(p_j))}\\
		&=\sum\limits_{i,j=1}^{N,N}{-p_ip_j\ln(p_i)}+\sum\limits_{i,j=1}^{N,N}{-p_ip_j\ln(p_j)}\\
		&=\sum\limits_{i=1}^{N}{-(\sum\limits_{j=1}^{N}{p_j})p_i\ln(p_i)}+\sum\limits_{j=1}^{N}{-(\sum\limits_{i=1}^{N}{p_i})p_j\ln(p_j)}\\
		&=\sum\limits_{i=1}^{N}{-p_i\ln(p_i)}+\sum\limits_{j=1}^{N}{-p_j\ln(p_j)}\\
		&=2\sum\limits_{i=1}^{N}{-p_i\ln(p_i)}=2I_1
	\end{align}
	If some characters are correlated, the entropy becomes\\
	\begin{align}
		I_2'&=\sum\limits_{i,j=1}^{N,N}{-p_ip_jG_{i,j}\ln(p_ip_jG_{i,j})}\\
		&=\sum\limits_{i,j=1}^{N,N}{-p_ip_jG_{i,j}(\ln(p_i)+\ln(p_j)+\ln(G_{i,j}))}\\
	\end{align}
	Notice that
	\begin{align}
		\sum\limits_{j=1}^{N}{p_ip_jG_{i,j}}=p_i\rightarrow\sum\limits_{j=1}^{N}{p_jG_{i,j}}=1
	\end{align}
	\begin{align}
		I_2'&=\sum\limits_{i=1}^{N}{-p_i\ln(p_i)}+\sum\limits_{j=1}^{N}{-p_j\ln(p_j)}+\sum\limits_{i,j=1}^{N,N}{-p_ip_jG_{i,j}\ln(G_{i,j})}\\
		I_2-I_2'&=\sum\limits_{i,j=1}^{N,N}{p_ip_jG_{i,j}\ln(G_{i,j})}\\
	\end{align}
	(c)\\
	The code written in C is shown below.
	\begin{lstlisting}[language={[ANSI]C},numbers=left,basicstyle=\ttfamily]
#include "stdio.h"
#include "math.h"
#define BUF_SIZE 1000
#define CHAR_NUM 256

int main()
{
  FILE* fp=fopen("./1.pdf","rb");
  int ch_cnt=0;
  int data_cnt=0;
  unsigned int ch_pool[CHAR_NUM];
  unsigned char buf[BUF_SIZE];
  float entropy=0.0;

  for(int i=0;i<CHAR_NUM;++i)
    ch_pool[i]=0;
  ch_cnt=fread(buf,sizeof(unsigned char),BUF_SIZE,fp);
  while(ch_cnt>0)
  {
    data_cnt+=ch_cnt;
    for(int i=0;i<ch_cnt;++i)
      ch_pool[buf[i]]+=1;
    ch_cnt=fread(buf,sizeof(unsigned char),BUF_SIZE,fp);
  }
  printf("Total=%d\n",data_cnt);
  for(int i=0;i<CHAR_NUM;++i)
  {
    float prob=(float)ch_pool[i]/(float)data_cnt;
    if(prob>0)
      entropy-=prob*log(prob);
  }
  printf("I=%f\n",entropy);
  printf("ln(256)=%f\n", log(256));
  fclose(fp);
  return 0;
}
	\end{lstlisting}
	in which 1.pdf is the pdf version of the Pathrina book, and the entropy result is 5.438327. This result is slightly smaller than the theorical maximum $\ln{256}=5.545177$.
\end{document}
>>>>>>> Stashed changes
