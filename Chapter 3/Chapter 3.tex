\documentclass{article}

\usepackage[left=1.5cm, right=1.5cm, top=3cm, bottom = 3cm]{geometry}

\usepackage{amsmath}
\usepackage{amsfonts}
\usepackage{amssymb}
\usepackage{graphicx}
\usepackage{float}
\usepackage{wrapfig}
\usepackage{latexsym}
\usepackage{hyperref}
\usepackage{feynmf}
\linespread{1.1}

%%%%%%%
%第三章习题安排:
%%%%%%%
%宋志坚 1,2,3,4
%宋盛雨央 10,20,30,40
%陈博文 11,21,31,41
%解放 12,22,32,42
%辜晨曦 13,23,33,43
%鲍亦澄 14,24,34,44
%蒋文韬 5,15,25,35
%李嘉琛 6,16,26,36
%颜公望 7,17,27,37
%张传坤 8,18,28,38
%王志凌 9,19,29,39

\author{SM-at-THU}
\title{\bf{Solutions to Pathria's Statistical Mechanics}\\Chapter 3}

\begin{document}
\maketitle

\section*{Problem 3.1}

\section*{Problem 3.2}

\section*{Problem 3.3}

\section*{Problem 3.4}

\section*{Problem 3.5}
	Since the Helmholtz free energy $A(N,V,T)$ has the property:
	\begin{equation*}
		A(\lambda N,\lambda V,T) = \lambda A(N,V,T)
	\end{equation*}
	Differentiate with respect to $\lambda$ and substitute $\lambda=1$ immediately yields
	\begin{equation*}
		N\left( \frac{ \partial A }{\partial N} \right)_{V,T}+V \left( \frac{ \partial A }{\partial V} \right)_{N,T} = A
	\end{equation*}

\section*{Problem 3.6}

\section*{Problem 3.7}

 \section*{Problem 3.9}
 	For an ideal monaomic gas,its heat capacity C would be 3R/2.While asume the whole progress is quasistatic,it would obey 
 	\begin{equation*}
		pV=RT
	\end{equation*}
	\begin{equation*}
		dU=-pdV+dQ=CdT
	\end{equation*}
	So we can get
	\begin{equation*}
		\frac{5}{2}pdV+\frac{3}{2}Vdp=dQ
	\end{equation*}
	For adiabatical process,dQ=0,so the ratio of the final pressure to initial pressure would be
	\begin{equation*}
		\frac{p_f}{p_i}=(1/2)^{5/3}
	\end{equation*}
	For the process with heat,the equation is difficult to solve,but naively thinking,for a process that the pressure doesn't change,it need heat to be added,so the final pressure would be higher than adiabatical process.
	
	
\section*{Problem 3.11}
Suppose $pV^{n} = C$, so the work done is
\begin{equation}
\Delta W = \int^{V_{2}}_{V_{1}} \frac{C}{V^{n}} dV = \frac{C}{n-1} (V^{1-n}_{2} - V^{1-n}_{1})
\end{equation}
The energy difference is given by
\begin{equation}
\Delta U = p_{2}V_{2} - p_{1}V_{1} = C (V^{1-n}_{2} - V^{1-n}_{1})
\end{equation}
Therefore, the heat absorbed is
\begin{equation}
\Delta Q =  C\frac{n-2}{n-1} (V^{1-n}_{2} - V^{1-n}_{1})
\end{equation}

\section*{Problem 3.12}
The Hamiltonian of the classical system can be written as:
\begin{equation}
H = \sum_{i}^{N} \frac{\mathbf{p}_i^2}{2m}+\sum_{i}^{N}U(\mathbf{x}_i)
\end{equation}
So the partition function of the system is:
\begin{eqnarray}
Q(\beta,N,V) &=& \frac{1}{N!h^{3N}}\int\prod_{i=1}^N d^3x_id^3p_i e^{-\beta H(x,p)}\nonumber\\
&=&\frac{1}{N!}\left[\left(\frac{2\pi m\beta^{-1}}{h^2}\right)^{3N/2}\int \prod_{i}d^3x_i e^{-\beta U(\mathbf{x}_i)}\right]\nonumber\\
\end{eqnarray}
So the Helmholtz potential is $A = -kT\ln Q$ and the entropy $S$ is the derivative of free energy:
\begin{eqnarray}
S &=& -\frac{\partial A}{\partial T}\nonumber\\
&=&-\frac{\partial}{\partial T}\left\{-kT\ln\left[\frac{1}{N!}\left(\frac{2\pi mkT}{h^2}\right)^{3N/2}\left(\int \prod_id^3x_i e^{-\beta U(\mathbf{x}_i)}\right)\right]\right\}\nonumber\\
&=&-\frac{\partial}{\partial T}\left\{-NkT\ln\left[\frac{1}{N}\left(\frac{2\pi mkT}{h^2}\right)^{3/2}\left(\int \prod_id^3x_i e^{-\beta U(\mathbf{x}_i)}\right)^{1/N}\right]-NkT\right\}\nonumber\\
&=&Nk\ln\left[\frac{1}{N}\left(\frac{2\pi mkT}{h^2}\right)^{3/2}\left(\int\prod_i d^3x_i e^{-\beta U(\mathbf{x}_i)}\right)^{1/N}\right]+\frac{3}{2}Nk + \frac{1}{T}\frac{\int \prod_i d^3x_i \sum_i U(\mathbf{x}_i)e^{-\beta U(\mathbf{x}_i)}}{\int \prod_i d^3x_i e^{-\beta U(\mathrm{x}_i)}} + Nk\nonumber\\
&=&\frac{5Nk}{2} + Nk\ln\left[\frac{1}{N}\left(\frac{2\pi mkT}{h^2}\right)^{3/2}\left(\int\prod_i d^3x_i e^{-\beta U(\mathbf{x}_i)}\right)^{1/N}\right] + \frac{\overline{U}}{T}\nonumber\\
&=&\frac{5Nk}{2} + Nk\ln\left[\frac{1}{N}\left(\frac{2\pi mkT}{h^2}\right)^{3/2}e^{\frac{\overline{U}}{NkT}}\left(\int\prod_i d^3x_i e^{-\beta U(\mathbf{x}_i)}\right)^{1/N}\right] \nonumber\\
&=&Nk\left\{\frac{5}{2}+ \ln \left[\frac{\overline{V}}{N}\left(\frac{2\pi mkT}{h^2}\right)^{3/2}\right]\right\}
\end{eqnarray}
Up till now we have shown the entropy of such a system. So if the potential energy is just a constant, the ``free volume'' is the common volume of classical ideal gas. 

Then consider about the hard sphere gas. The potential energy is:
$$
U(\mathbf{x}_i)=\left\{\begin{array}{ll}
0 &|\mathbf{x}_i-\mathbf{x}_j| > D\\
\infty &|\mathbf{x}_i-\mathbf{x}_j| < D
\end{array}\right.
$$
It is obvious that the average of potential energy is $\overline{U} = 0$, so the free volume is
\begin{eqnarray}
\overline{V}^N&=&\int \prod_i d^3x_i\,e^{-\beta U(\mathbf{x}_i)}\nonumber\\
&=&\int d^3x_N\int d^3x_{N-1}\cdots \int d^3x_1 e^{-\beta U(\mathbf{x}_i)}\nonumber\\
&=&V\left(V-\frac{4\pi}{3}D^3\right)\left(V-2\cdot\frac{4\pi}{3}D^3\right)\cdots\left(V-\frac{N-1}{3}4\pi D^3\right)
\end{eqnarray}
Define $v_0 = \pi D^3/6$ is the volume a sphere, so the gas-law will be:
\begin{eqnarray}
P &=& \frac{NkT}{\overline{V}}\frac{\partial \overline{V}}{\partial V}\nonumber\\
&=& kT\left(\frac{1}{V}+\frac{1}{V-8v_0}\cdots \frac{1}{V+8(N-1)v_0}\right)\nonumber\\
&\simeq & kT\left(\frac{N+N^2\frac{4v_0}{V}}{V}\right)\nonumber\\
&=& kT\frac{N}{V\frac{1}{1+4Nv_0/V}}\nonumber\\
&\simeq &\frac{NkT}{V-4Nv_0}
\end{eqnarray}
This result is the same as we have seen in Problem 1.4.


\section*{Problem 3.15} % (fold)
\label{sec:problem_3_15}

	We have $Q_1(V,T)=\int g(\epsilon) e^{-\beta \epsilon}d \epsilon $. For 3-D extreme relativistic gas, $\epsilon =pc $, hence we have
	\begin{gather*}
		g(p)dp=\frac{V}{h^3}4 \pi p^2dp = \frac{4 \pi V }{h^3} \frac{\epsilon^2}{c^2} \frac{d \epsilon}{c} = g(\epsilon)d \epsilon\\
		\therefore g(\epsilon)=\frac{4 \pi V}{(hc)^3} \epsilon^2\\
		\therefore Q_1(V,T)=\int^{\infty}_0 g(\epsilon)d \epsilon=\frac{4 \pi V}{(hc)^3} \int^{\infty}_0 \epsilon^2 e^{-\beta \epsilon}d \epsilon = 8 \pi V \left( \frac{kT}{hc} \right)^3
	\end{gather*}
	
	$\therefore$for $N$ molecules,
	\begin{equation*}
		Q_N(V,T)= \frac{1}{N!}\left\{ 8 \pi V \left( \frac{kT}{hc} \right)^3\right\}
	\end{equation*}

	From $Q_N(V,T)$, it's easy to calculate:
	\begin{gather*}
		P=\frac{1}{\beta} \frac{\partial Q}{\partial V} = \frac{N}{V}kT\\
		U = -\frac{1}{Q} \frac{\partial Q}{\partial \beta}=3NkT\\
		\gamma = \frac{4}{3}
	\end{gather*}

% section problem_3_15 (end)

\section*{Problem 3.19}
	\begin{equation*}
		<\frac{dG}{dt}>=<\sum p_i \frac{dq_i}{dt}>+<\sum q_i \frac{dp_i}{dt}>=0
	\end{equation*}
	Above equation has used equation (3.7.5) and equation (3.7.6).
	The equation (3.7.5) and equation(3.7.6)  both come from (3.7.2),so validity of one equation implies another's.
	

\section*{Problem 3.21}
\noindent (a) Classically, the harmonic equation of motion leads to $x = A \sin \omega t$. As a result, the kinetic energy and potential energy will be $m \omega^{2} A^{2} \cos^{2} \omega t /2$ and $m \omega^{2} A^{2} \sin^{2} \omega t /2$ respectively. Average them it's easy to see that $\bar{K} = \bar{U} =m \omega^{2} A^{2}/4$.\\
Quantum-mechanically, $\psi = \sum_{n} c_{n} \psi_{n}$ where $\psi_{n}$ is the \emph{n}-th Hermitian polynomial. Using the recursive relations, we have
\begin{align}
&\bar{K} = -\frac{\hbar^{2}}{2m} \sum_{n} |c_{n}|^{2} \int \psi^{*} \frac{d^{2}}{dx^{2}} \psi dx = \sum_{n} |c_{n}|^{2} \frac{\hbar \omega (2n+1)}{4} = \frac{1}{2} \sum_{n} |c_{n}|^{2} E_{n}\\
&\bar{U} = \frac{m \omega^{2}}{2} \sum_{n} |c_{n}|^{2} \int \psi^{*} x^{2} \psi dx= \sum_{n} |c_{n}|^{2} \frac{\hbar \omega (2n+1)}{4} = \frac{1}{2} \sum_{n} |c_{n}|^{2} E_{n}
\end{align}
\noindent (b) In Bohr-sommerfeld model, a quantized orbits are hypothesized, namely $m_{e}vr = n \hbar$. In the \emph{n}-th orbit, the total energy is $E_{n} = - Z^{2}k^{2}e^{4}m_{e}/2 \hbar^{2} n^{2}$. The radius of which is $r_{n} = n^{2} \hbar^{2} / Zke^{2}m_{e}$. By a naive calculation $\bar{U} = - Z^{2}k^{2}e^{4}m_{e}/ \hbar^{2} n^{2}$ and $\bar{T} = Z^{2}k^{2}e^{4}m_{e}/2 \hbar^{2} n^{2}$.\\
In the Schroedinger hydrogen atom, $\psi_{nlm} = R_{nl}(r) Y_{lm}(\theta, \phi)$. The kinetic energy is given by
\begin{eqnarray}
\bar{T} &=& \frac{\hbar^{2}}{2m} \int \psi^{*}_{nlm} (\frac{d^{2}}{dr^{2}} + \frac{2}{r} \frac{d}{dr} - \frac{l(l+1)}{r^{2}})\psi_{nlm} r^{2} \sin \theta dr d\theta d\phi \notag \\
&=& \frac{\hbar^{2}}{2m} \int R_{nl}(r) (\frac{1}{n^{2}a^{2}}) R_{nl}(r) r^{2}dr \notag \\
&=& \frac{e^{2}}{2a n^{2}}
\end{eqnarray}
so $\bar{U} = - e^{2}/a n^{2}$. \emph{a} is the Bohr radius.\\
\noindent (c) This is also a central force case. The results are quite identical to (b).

\section*{Problem 3.22}
Anharmonic Oscillator.

This anharmonic oscillator has the Hamiltonian:
$$
H = \frac{p^2}{2m} + \frac{1}{4}kx^4
$$
So the canonical partition function of the system is:
\begin{equation}
Q = \frac{1}{h}\int dpdx\,e^{-\beta\left(\frac{p^2}{2m} + \frac{1}{4}kx^4\right) }
\end{equation}
Use the ``equipartition theorem'', we can get the following result:
\begin{equation}
\Big{\langle} x \frac{\partial H}{\partial x}\Big{\rangle} =kT
\end{equation}
Thus because $\partial H/\partial x = kx^3$, we can get
$$
x\frac{\partial H}{\partial x} = kx^4 = 4V
$$
So the expectation value of the potential is $\langle V\rangle = kT/4$. For the same reason, we can get the mean value of the kinetic energy:
\begin{equation}
\langle K\rangle = \frac{1}{2}\Big{\langle} p\frac{\partial H}{\partial p}\Big{\rangle} = \frac{kT}{2}
\end{equation}
So clearly we can get $\langle K\rangle = 2\langle V\rangle$.

\section*{Problem 3.29}
	I can't solve this problem.The intergral of the unharnomic terms in the partition fuction is infinite.
	
\section*{Problem 3.31}
``Partition function'' for single particle is
\begin{equation}
Q_{1} = 1 + e^{-\varepsilon/kT}.
\end{equation}
So a list of quatities can be obtained:
\begin{align}
&Q_{N} = (1 + e^{-\varepsilon/kT})^{N} \\
&A = - NkT \ln (1 + e^{-\varepsilon/kT}) \\
&\mu = - kT \ln (1 + e^{-\varepsilon/kT}) \\
&p=0 \\
&S = Nk \ln (1 + e^{-\varepsilon/kT}) + \frac{N \varepsilon}{T} \frac{e^{-\varepsilon/kT}}{1+e^{-\varepsilon/kT}} \\
&U =  N \varepsilon \frac{e^{-\varepsilon/kT}}{1+e^{-\varepsilon/kT}} \\
&C_{p} = C_{V} =  \frac{N \varepsilon^{2} e^{-\varepsilon/kT}}{kT^{2} (1+e^{-\varepsilon/kT})^{2}}
\end{align}
This specific heat is sometimes referred to \emph{Schottky anomaly}. 

\section*{Problem 3.32}
\paragraph{(a)} Since the distribution is given by canonical distribution, the probabilities are:
$$
p_i = Q^{-1}g_ie^{-\beta \epsilon_i}
$$
and the entropy should be:
\begin{eqnarray}\label{entropy}
S &=& -k\left[p_1\ln(p_1/g_1)+p_2\ln(p_2/g_2)\right]\nonumber\\
&=& -k\left[\frac{g_1e^{-\beta \epsilon_1}}{Q}\ln \frac{e^{-\beta\epsilon_1}}{Q}+\frac{g_2e^{-\beta \epsilon_2}}{Q}\ln \frac{e^{-\beta\epsilon_2}}{Q}\right]\nonumber\\
&=& k\ln Q + \frac{1}{T}\frac{g_1\epsilon_1e^{-\beta\epsilon_1}+g_2\epsilon_2e^{-\beta\epsilon_2}}{Q}\nonumber\\
&=&k\ln g_1 + k\ln\left(1+\frac{g_2}{g_1}e^{-x}\right)+\frac{1}{T}\frac{g_2(\epsilon_2-\epsilon_1)e^{-\beta\epsilon_2}}{Q}\nonumber\\
&=&k\left[\ln g_1 + \ln\left(1+\frac{g_2}{g_1}e^{-x}\right)+\frac{g_2e^{-\beta\epsilon_2}x}{Q}\right]\nonumber\\
&=&k\left[\ln g_1 + \ln\left(1+\frac{g_2}{g_1}e^{-x}\right)+\frac{x}{1+\frac{g_1}{g_2}e^x}\right]
\end{eqnarray}
When $g_1=g_2=1$, the situation is the same as Fermi oscillator with energy $0$ and $\epsilon_2-\epsilon_1$.
\paragraph{(b)}
The entropy is the derivative of the free energy, so we can get the entropy by the following process:
\begin{eqnarray}
S &=& -\frac{\partial A}{\partial T}\nonumber\\
&=&\frac{\partial}{\partial T}\left\{kT\ln Q\right\}\nonumber\\
&=&k\ln Q +\frac{1}{T}\frac{g_1\epsilon_1e^{-\beta\epsilon_1}+g_2\epsilon_2e^{-\beta\epsilon_2}}{Q}\nonumber\\
&=&k\left[\ln g_1 + \ln\left(1+\frac{g_2}{g_1}e^{-x}\right)+\frac{x}{1+\frac{g_1}{g_2}e^x}\right]
\end{eqnarray}
which is the same as we get in (a).
\paragraph{(c)}
Clearly from equation (\ref{entropy}), when temperature is $T=0$, the entropy will be:
\begin{equation}
S = \lim_{x\rightarrow +\infty}k\left[\ln g_1 + \ln\left(1+\frac{g_2}{g_1}e^{-x}\right)+\frac{x}{1+\frac{g_1}{g_2}e^x}\right]=k\ln g_1
\end{equation}
From the distribution of canonical ensemble, we know that when the temperature is $T=0$, the system will stay on the ground state. Since the ground is g-fold degenerate, there are $g_1$ possible states. So the entropy is $S = k\ln g_1$.

\section*{Problem 3.39}
	By using equation (3.9.18) we could get
	\begin{equation*}
		Q=\sum_{m=-1/2}^{1/2}\exp(\beta g \mu_b mH)=\exp(-\beta g \mu_b H/2)(1+\exp(\beta g \mu_b H))
	\end{equation*}
	The mean magnetic moment is
	\begin{equation*}
		M=\frac{N}{\beta}\frac{\partial}{\partial H}\ln Q=\frac{1}{2}N\mu_b g\frac{1-\exp(-\beta g \mu_b H)}{1+\exp(-\beta g \mu_b H)}
	\end{equation*}
	While the number of parallel atoms $N_+$ and antiparallel $N_-$ satisfied that
	\begin{equation}
		\left\{
		\begin{aligned}
		\overset{.} N_+ +N_- =N\\
		(N_+ -N_-) g \mu_b J =M\\
		J=1/2
		\end{aligned}
		\right.
	\end{equation}
	So we could get the answer
	\begin{equation}
		\left\{
		\begin{aligned}
		\overset{.} 
		{N_+}/{N}=\frac{1}{1+\exp(-\beta g \mu_b H)}\\
		{N_-}/{N} =\frac{\exp(-\beta g \mu_b H)}{1+\exp(-\beta g \mu_b H)}\\
		\end{aligned}
		\right.
	\end{equation}
	Acoording to the given situation,flux density 0.1 $weber/m^2$ and temperature of 1000K,the respective fractions are
	\begin{equation}
		\left\{
		\begin{aligned}
		\overset{.} 
		{N_+}/{N}=50.00168\%\\
		{N_-}/{N} =49.99832\%\\
		\end{aligned}
		\right.
	\end{equation}
	
\section*{Problem 3.41}
The equilibrium temperature will be positive, since the energy of the whole system is not bounded from above. This case is a bit like the spin and lattice case. For the subsystem of spins, its energy is bounded from above, so it is possible to attain a negative temperature. While the subsystem of lattice, i.e. ideal gas in this problem, only has positive temperature. The whole system doesn't have a energy limit, so the temperature will only be positive. And energy may flow from the spin subsystem to the ideal gas.

\section*{Problem 3.42}
Paramagnetic system.

For a given energy $E$, we can know that:
\begin{eqnarray}
E &=& \mu_B H (N_{\uparrow}-N_\downarrow)\\
N &=& N_\uparrow + N_\downarrow
\end{eqnarray}
So the occupying number of up(down)-spin is
$$
N_\uparrow = \frac{1}{2}\left(N+\frac{E}{\mu_B H}\right)\quad N_\downarrow = \frac{1}{2}\left(N-\frac{E}{\mu_B H}\right)
$$
And the number of the possible states will be:
\begin{equation}
\Omega(N,E) = \mathrm{C}_{N}^{N_\uparrow}=\frac{N!}{N_\uparrow !N_\downarrow!}
\end{equation}
So the entropy in micro canonical ensemble representation is:
\begin{eqnarray}
S &=& k\ln\Omega(E,N) \nonumber\\
&=& Nk\ln N -N_\uparrow k \ln N_\uparrow - N_\downarrow k \ln N_\downarrow\nonumber\\
&=& Nk\ln N -k\frac{N\mu_B H +E}{2\mu_B H}\ln \frac{N\mu_B H +E}{2\mu_B H}-k\frac{N\mu_B H -E}{2\mu_B H}\ln\frac{N\mu_B H -E}{2\mu_B H}
\end{eqnarray}
This result is the same as (3.10.9) in Pathria's Book. Then the temperature:
\begin{eqnarray}
\frac{1}{T} &=& \frac{\partial S}{\partial E}\nonumber\\
&=&-\frac{k}{2\mu_B H}\ln \frac{N\mu_B H +E}{2\mu_B H}-\frac{k}{2\mu_B H}+\frac{k}{2\mu_BH}\ln\frac{N\mu_B H-E}{2
mu_B H}+\frac{k}{2\mu_BH}\nonumber\\
&=&\frac{k}{2\mu_BH}\ln\left(\frac{N\mu_BH-E}{N\mu_BH+E}\right)
\end{eqnarray}
And this result is also the same as equation (3.10.8).

\end{document}